\documentclass{beamer}
\usepackage[T1]{fontenc}
\usepackage[utf8]{inputenc}
\usepackage{listings}
\usepackage{caption}
\usepackage{amssymb}
\usepackage{upgreek}
\usepackage{mathtools}
\usepackage{graphicx}
\graphicspath{ {./images/} }
\usepackage{tikz}
\usetikzlibrary{shapes.callouts,shadows, calc}

\usepackage[dvipsnames]{xcolor}
%colors for lean syntax
\definecolor{keywordcolor}{rgb}{0.7, 0.1, 0.1}   % red
\definecolor{tacticcolor}{rgb}{0.0, 0.1, 0.6}    % blue
\definecolor{commentcolor}{rgb}{0.4, 0.4, 0.4}   % grey
\definecolor{symbolcolor}{rgb}{0.0, 0.1, 0.6}    % blue
\definecolor{sortcolor}{rgb}{0.1, 0.5, 0.1}      % green
\definecolor{attributecolor}{rgb}{0.7, 0.1, 0.1} % red

\def\lstlanguagefiles{lstlean.tex}
% set default language
\lstset{language=lean}

\usetheme[compress]{Berlin}
\usecolortheme{default}

\captionsetup{labelformat=empty}

\newcommand{\bA}{\mathbb{A}}
\newcommand{\bB}{\mathbb{B}}
\newcommand{\bC}{\mathbb{C}}
\newcommand{\bD}{\mathbb{D}}
\newcommand{\bE}{\mathbb{E}}
\newcommand{\bF}{\mathbb{F}}
\newcommand{\bG}{\mathbb{G}}
\newcommand{\bH}{\mathbb{H}}
\newcommand{\bI}{\mathbb{I}}
\newcommand{\bJ}{\mathbb{J}}
\newcommand{\bK}{\mathbb{K}}
\newcommand{\bL}{\mathbb{L}}
\newcommand{\bM}{\mathbb{M}}
\newcommand{\bN}{\mathbb{N}}
\newcommand{\bO}{\mathbb{O}}
\newcommand{\bP}{\mathbb{P}}
\newcommand{\bQ}{\mathbb{Q}}
\newcommand{\bR}{\mathbb{R}}
\newcommand{\bS}{\mathbb{S}}
\newcommand{\bT}{\mathbb{T}}
\newcommand{\bU}{\mathbb{U}}
\newcommand{\bV}{\mathbb{V}}
\newcommand{\bW}{\mathbb{W}}
\newcommand{\bX}{\mathbb{X}}
\newcommand{\bY}{\mathbb{Y}}
\newcommand{\bZ}{\mathbb{Z}}

\newcommand{\pr}[2]{\text{pr}_{#1}(#2)} %projection
\newcommand{\compl}[1]{#1^{c}} %complement
\newcommand{\id}{\text{id}} %identity

%Information to be included in the title page:
\title{Formalisation of CW complexes}
\author{Hannah Scholz}
\institute[MI]{Mathematical Institute of the University of Bonn}
\date[\today]{\today \\ \vspace{1cm} \footnotesize{Joint work with and supervised by Prof.\ Floris van Doorn}}
%\date{20.09.2024}



\begin{document}

\frame{\titlepage}

\begin{frame}
\frametitle{Why CW complexes?}
\begin{itemize}
  \item Very general class of spaces
  \begin{itemize}
    \item Examples of CW complexes: $\bR^n$, $S^n$, $\bC\bP^n$, $\bR\bP^\infty$
    \item Homotopy type of CW complexes: differentiable manifolds
    \item Not a CW complex: hedgehog space 
    \item Not homotopy equivalent to a CW complex: Hawaiian earring
  \end{itemize}
  \begin{figure}
        \begin{minipage}[b]{0.25\linewidth}
            \centering
            \includegraphics[width=\textwidth]{images/hedgehog_space.png}
            \caption{hedgehog space}
        \end{minipage}
        \hspace{2cm}
        \begin{minipage}[b]{0.25\linewidth}
            \centering
            \includegraphics[width=\textwidth]{images/hawaiian_earring.png}
            \caption{Hawaiian earring}
        \end{minipage}
    \end{figure}
\end{itemize}
\end{frame}

\begin{frame}
\frametitle{Why CW complexes?}
\begin{itemize}
  \item A lot of strong results about CW complexes
\end{itemize}
\begin{theorem}[Whitehead theorem, 1949]
A continuous map between two CW complexes that induces isomorphisms on all homotopy groups is a homotopy equivalence.
\end{theorem}
\begin{theorem}[Cellular homology]
Let $X$ be a CW complex. Then the cellular and singular homology of $X$ agree.
\end{theorem}
\end{frame}

\begin{frame}
\frametitle{Intuition: What is a CW complex?}
\begin{itemize}
  \item Glue $n$-cells (i.e.\ continuous images of $n$-discs) together along their boundaries
\end{itemize}
\pause
\begin{figure}
        \begin{minipage}[b]{0.25\linewidth}
            \centering
            \includegraphics[width=\textwidth]{images/zerocells.png}
            \caption{$0$-cells}
        \end{minipage}
        \hspace{0.5cm}
        \pause
        \begin{minipage}[b]{0.25\linewidth}
            \centering
            \includegraphics[width=\textwidth]{images/onecells.png}
            \caption{$1$-cells}
        \end{minipage}
        \hspace{0.5cm}
        \pause
        \begin{minipage}[b]{0.25\linewidth}
            \centering
            \includegraphics[width=\textwidth]{images/twocells.png}
            \caption{$2$-cells}
        \end{minipage}
    \end{figure}
\end{frame}

\begin{frame}
\frametitle{Examples: What is a CW complex?}
\begin{figure}
    \begin{minipage}[b]{0.25\linewidth}
        \centering
        \includegraphics[width=\textwidth]{images/interval.png}
        \caption{Interval}
    \end{minipage}
    \hspace{0.5cm}
    \pause
    \begin{minipage}[b]{0.5\linewidth}
        \centering
        \includegraphics[width=\textwidth]{images/realline.png}
        \caption{Real line}
    \end{minipage}
    \hspace{0.5cm}
\end{figure}
\pause
\begin{figure}
  \includegraphics[width=0.4\textwidth]{images/sphere.png}
        \caption{$2$-sphere}
\end{figure}
\end{frame}

\begin{frame}
\frametitle{Definition: What is a CW complex?}
\small  
Let $X$ be a Hausdorff space. 
  A \emph{CW complex} on $X$ consists of a family of indexing sets $(I_n)_{n \in \bN}$ and a family of continuous maps $(Q_i^n \colon D^n \to X)_{n \in \bN, i \in I_n}$ called \emph{characteristic maps} with the following properties: 
  \setbeamertemplate{enumerate items}[default]
  \begin{enumerate}[(i)]
      \item $\restrict{Q_i^n}{\interior{D^n}} \colon \interior{D^n} \to Q_i^n(\interior{D^n})$ is a homeomorphism for every $n \in \bN$ and $i \in I_n$. We call $\openCell{n}{i} \coloneq Q_i^n(\interior{D^n})$ an \emph{(open) $n$-cell} and $\closedCell{n}{i} \coloneq Q_i^n(D^n)$ a \emph{closed $n$-cell}.
      \item Two different open cells are disjoint.
      \item For each $n \in \bN$ and $i \in I_n$ the \emph{cell frontier} $\cellFrontier{n}{i} \coloneq Q_i^n(\boundary D^n)$ is contained in the union of a finite number of closed cells of a lower dimension.
      \item A set $A \subseteq X$ is closed if the intersections $A \cap \closedCell{n}{i}$ are closed for all $n \in \bN$ and $i \in I_n$.
      \item The union of all closed cells is $X$.
    \end{enumerate}
\end{frame}

\begin{frame}
\frametitle{Lean: What is a CW complex?}
\includegraphics[width=1.05\textwidth]{images/CWComplexinLean.png}
\end{frame}

\begin{frame}[fragile]
\frametitle{What has been done in Lean?}
By other people (that I am aware of): 
  \begin{itemize}
    \item[\textcolor{Green}{\textbullet}] Categorical definition \lstinline|TopCat.RelativeCWComplex| by Jiazhen Xia and Elliot Dean Young and refactored by Joël Riou: in Mathlib
    \item[\textcolor{YellowOrange}{\textbullet}] Equivalence of the definitions by Robert Maxton: PRs
  \end{itemize}
  \begin{itemize}
    \item[\textcolor{Green}{\textbullet}] Definition and basic properties: In mathlib
    \item[\textcolor{Yellow}{\textbullet}] Finiteness notions on CW Complexes: PR
    \item[\textcolor{YellowOrange}{\textbullet}] Subcomplexes: Done
    \item[\textcolor{YellowOrange}{\textbullet}] Basic constructions: Done 
    \item[\textcolor{Yellow}{\textbullet}] KSpaces: PR
    \item[\textcolor{YellowOrange}{\textbullet}] Additional Lemmas: Done
    \item[\textcolor{YellowOrange}{\textbullet}] Product: Done
    \item[\textcolor{Orange}{\textbullet}] Examples: Needs clean-up
    \item[\textcolor{Orange}{\textbullet}] Maps: Work in progress
    \item[\textcolor{Red}{\textbullet}] Quotients: To be done (by me)
    \item[\textcolor{BrickRed}{\textbullet}] Equivalence to other definition: To be done (probably not by me)
  \end{itemize}
\end{frame}


\end{document}