\section{Definition of CW complexes}

\todo[plan]{
In this section I should first describe mathematically what a CW complex is. 
Then I show the Lean definition and explain why/how
Maybe this can also have proofs of basic lemmas (e.g. a CW complex is closed in its ambient space)}

A \emph{CW complex} is a topological space that can be constructed by glueing images of closed disc of different dimensions together along the images of their boundaries. 
These images of closed discs in the CW complex are called \emph{cells}.
To specify that a cell is the image of an $n$-dimensional disc, one can call it an $n$-cell.
The cells up to dimension $n$ make up what is called the \emph{$n$-skeleton}.
In a relative CW complex these discs can additionally be attached to a specified base set. 

The different definitions of CW complexes present in the literature can be broadly categorized into to approaches: firstly there is the ``classical'' approach that sticks closely in style to Whitehead's original definition in \cite{Whitehead2018}.
This definition assumes the cells to all lie in one topological space and then describes how the cells interact with each other and the space.
Secondly, there is a popular approach that is more categorical in nature. 
In this approach the skeletons are regarded as different spaces and the definition describes how to construct the $n+1$-skeleton from the $n$-skeleton. 
The CW complex is then defined as the colimit of the skeletons. 

At the start of this project neither of the approaches had been formalized in Lean. 
The authors chose to proceed with the former approach for the following reasons: 