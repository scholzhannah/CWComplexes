\section{Finiteness notions}

\todo[comment, inline]{Should this be a subsection in the definition section instead?}

There are three important finiteness notions on CW complexes. 
We say that a CW complex is \emph{of finite type} if there are only finitely many cells in each dimension. 
We call it \emph{finite dimensional} if there is an $n$ such that the complex equals its $n$-skeleton.
Finally, it is said to be \emph{finite} is it is both finite dimensional and of finite type. 
In Lean these definitions take the following form: 

\begin{lstlisting}[frame=single]
class RelCWComplex.FiniteDimensional.{u} {X : Type u} [TopologicalSpace X] 
    (C : Set X) {D : Set X} [RelCWComplex C D] : Prop where
  eventually_isEmpty_cell : ∀ᶠ n in Filter.atTop, IsEmpty (cell C n)

class RelCWComplex.FiniteType.{u} {X : Type u} [TopologicalSpace X] (C : Set X) 
    {D : Set X} [RelCWComplex C D] : Prop where
  finite_cell (n : ℕ) : Finite (cell C n)

class RelCWComplex.Finite {X : Type*} [TopologicalSpace X] (C : Set X) 
    {D : Set X} [RelCWComplex C D] extends FiniteDimensional C, FiniteType C
\end{lstlisting}

When defining a CW complex of finite type, we can add a condition stating that the type of cells in each dimension is finite and relax the condition \lstinline|mapsTo| of Figure \ref{def:cwcomplex} to be

\begin{lstlisting}[frame=single]
mapsTo : ∀ (n : ℕ) (i : cell n), MapsTo (map n i) (sphere 0 1) (D ∪ ⋃ (m < n) (j : cell m), map m j '' closedBall 0 1)
\end{lstlisting}

When constructing a finite CW complex, we can again add conditions stating that the type of cells in each dimension is finite and that starting at a large enough dimension it is empty.
In exchange, we can drop the condition \lstinline|closed'| of Figure \ref{def:cwcomplex} and modify the condition \lstinline|mapsTo| in the way described above. 
We provide constructors for both of these situations.

We then show that a CW complex is finite iff it is compact and that a compact subset of a CW complex is contained in a finite subcomplex.