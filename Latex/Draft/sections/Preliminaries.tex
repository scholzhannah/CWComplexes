\section{Preliminaries}

\subsection{Lean and \mathlib}\label{sub:leanandmathlib}

Lean and \mathlib make use of \emph{typeclasses} to provide definitions on various types with potentially different behaviour.
For example, \mathlib has a general notion of a topological space on an arbitrary type.
One can then provide specific \emph{instances} of a typeclass, for example the metric topology on the reals, or the subspace topology on a subtype of \lstinline|X|, assuming that \lstinline|X| has a topology. 
We write \lstinline|[TopologicalSpace X]| to assume that a space \lstinline|X| has a topology.
Additionally, there are forgetful instances, e.g. every metric space is a topological space. % it doesn't matter that this is probably a composition of instances in Mathlib.
Lean uses \emph{typeclass inference} to search the graph of instances to find the required instances.
More about typeclasses in Lean can be found in \cite{Selsam2020} and \cite{Baanen2025}.

% There are different ways to modify the behaviour of typeclass inference for specific typeclasses.
% In the definition of CW complexes, we tag certain parameters of a class with \lstinline|outParam|.
By default, Lean's typeclass inference algorithm requires all parameters to be known before it searches for an instance. This behaviour can be modified by marking certain parameters with \lstinline|outParam|, which means that typeclass inference will search for an instance even if the out-parameters are not yet known.
Furthermore, typeclass inference will not consider the values of \lstinline|outParam| parameters, and will search for the first instance where all parameters that are not marked as \lstinline|outParam| unify (up to definitional equality).
Therefore, an argument should only be marked as an \lstinline|outParam| if for every combination of the parameters not marked as \lstinline|outParam| there is at most one instance.
For example, the class \lstinline|Membership α β| states that for \lstinline|a : α| and \lstinline|b : β| we have specified the notation \lstinline|a ∈ b|. 
The first parameter is marked as an \lstinline|outParam|, since it is uniquely determined by the second parameter. Typical instances include \lstinline|Membership α (List α)| and \lstinline|Membership α (Set α)|.

Another technical detail of Lean that we will manipulate is reducibility.
The reducibility of a definition describes what processes are allowed to unfold it. 
Definitions in \mathlib are by default \lstinline|semireducible|, meaning they are unfolded for basic checks like definitional equality but not for more time intensive operations like typeclass inference.
However, sometimes we do want definitions to be unfolded in typeclass inference. 
To achieve this behaviour, we need to change the reducibility setting to \lstinline|reducible| which is done by writing \lstinline|abbrev| instead of \lstinline|def|. 
\mathlib, for example has the abbreviation \lstinline|unitInterval| for the interval \lstinline|Set.Icc 0 1 : Set ℝ|.
\todo[inline]{Better example? I think this one only exists to be able to provide notation? Would Path.Homotopy be better?}

More on \lstinline|outParam| and \lstinline|abbrev| can be found in \cite{LeanReference2025}.

Like a lot of programming languages, Lean makes use of namespaces to sort its declarations and reduce the length of names when namespaces are opened. 
One can prevent this shortening of names by specifying a declaration to be \lstinline|protected|. 
This ensures that the declaration is still available but makes using it less convenient. 
The modifier is frequently used to prevent naming conflicts in namespaces that might be opened together and do discourage the use of certain declarations in favour of more idiomatic ones. 

\subsection{Preliminary Mathematics in Lean}\label{sub:mathinlean}

In \mathlib, we write \lstinline|TopologicalSpace X| to say that a type \lstinline|X| has a topology. 
\lstinline|IsOpen A| and \lstinline|IsClosed A| for a set \lstinline|A| in \lstinline|X| assert openness and closedness. 
For a map \lstinline|f : X → Y| between topological spaces continuity is described as \lstinline|Continuous f| or \lstinline|ContinuousOn A| for continuity on a set. 
To specify that a topological space \lstinline|X| is Hausdorff one writes \lstinline|T2Space X|.

A partial bijection between two types \lstinline|X| and \lstinline|Y| has type \lstinline|PartialEquiv X Y| and is made up of two total functions \lstinline|X → Y| and \lstinline|Y → X|, a set in \lstinline|X| called the \emph{source}, a set in \lstinline|Y| called the \emph{target} and proofs that that the target is mapped to the source and vice versa and that the two maps are inverse to each other on both the source and target.

A \emph{completely distributive lattice} is a complete lattice in which even infinite joins and meets distribute over each other. 
This differentiates it from complete distributive lattices. 
In \mathlib, one writes \lstinline|CompletelyDistribLattice α| to provide a completely distributive lattice structure on \lstinline|α|. 
We furthermore write \lstinline|x ⊔ y| for the binary and \lstinline|⨆ (i : ι), x i| for the indexed join, and \lstinline|x ⊓ y| for the binary and \lstinline|⨅ (i : ι), x | for the indexed meet.\todo[comment]{Explain the definition in more detail? Here or below?}