\section{Products}\label{sec:products}

In general, the product of two CW complexes is not necessarily a CW complex because the weak topology of the CW complex might not match the product topology.
A counterexample was first provided by \citeauthor{Dowker1952} in \cite{Dowker1952}.

In order to achieve the correct topology on the product we need to turn it into a compactly generated space which we will discuss in the next subsection.

\subsection{Compactly generated spaces}

The name ``\emph{compactly generated space}'' (or sometimes ``\emph{k-space}'') is used for different notions in the literature.
Firstly, it can refer to a space with a topology that is coherent with its compact subsets, i.e.\ a set is closed if and only if its intersection with every compact subset is closed in that subset. 
Secondly, it can refer to a space with a topology determined by continuous maps from compact Hausdorff spaces, i.e.\ a set is closed if and only if its preimage under every continuous map from a compact Hausdorff map is closed. 
Thirdly, it can refer to a space with a topology coherent with its compact Hausdorff subspaces, i.e.\ a set is closed if and only if its intersection with every compact Hausdorff subspace is closed in that subspace. 
We believe that the classification into these three notions was first done by \citeauthor{Wikipedia2025} in \cite{Wikipedia2025}.

While these three notions agree for Hausdorff spaces, in the general case, the first is the weakest and the third the strongest. 
When starting this formalization, the second version was already in \mathlib as \lstinline|CompactlyGeneratedSpace|; the two other versions had not been formalized. 
We intended to follow the construction of the product presented in \cite{Hatcher2002} which uses the first version of compactly generated spaces. 
Since we assume our ambient space to be Hausdorff, we could have just translated the proof to use the version already in \mathlib. 
Instead, we decided to formalize the first version and named it \lstinline|CompactlyCoherentSpace|.
This name was suggested by Steven Clontz. 
We will also refer to the mathematical notion as a \emph{compactly coherent space} going forward.

In \mathlib, the definition is the following: 

\begin{lstlisting}[frame=single]
class CompactlyCoherentSpace (X : Type*) [TopologicalSpace X] : Prop where
  isCoherentWith : IsCoherentWith (X := X) {K | IsCompact K}
\end{lstlisting}

which uses the already pre-existing structure \lstinline|IsCoherentWith| that is defined as: 

\begin{lstlisting}[frame=single]
structure IsCoherentWith (S : Set (Set X)) : Prop where
  isOpen_of_forall_induced (u : Set X) : 
    (∀ s ∈ S, IsOpen ((↑) ⁻¹' u : Set s)) → IsOpen u
\end{lstlisting}\todo[comment]{Credit author of IsCoherentWith?}

Here, \lstinline|↑ : Set s → Set X| is the natural inclusion from \lstinline|Set s| into \lstinline|Set X| where \lstinline|s : Set X|. 
Thus the condition \lstinline|isOpen_of_forall_induced| states that for all subsets \lstinline|u| of \lstinline|X| and all elements \lstinline|s| of the collections of subsets \lstinline|S|, if the preimage of \lstinline|u| under \lstinline|↑ : Set s → Set X|, i.e.\ the intersection of \lstinline|u| with \lstinline|s|, is open in \lstinline|s|, then \lstinline|u| is open in \lstinline|X|.

We first show that this definition is equivalent to the one characterizing closedness which we stated at the beginning of the subsection. 
\mathlib already had the proofs for two examples of compactly coherent spaces: sequential spaces (which for example include metric spaces) and weakly locally compact spaces. 
Lastly, we show that \lstinline|CompactlyCoherentSpace| is a weaker notion of \lstinline|CompactlyGeneratedSpace| but that the two agree assuming the space is Hausdorff.

\todo[inline, color=green]{Anatole Dedecker refactored a lot of this section (k-ification) for me in the review process. How do I credit that?}
Next, we want to provide a way to turn any topological space into a compactly coherent space. 
This operation is typically referred to as \emph{k-ification}. 
We will call it \emph{compact coherentification} corresponding to our naming of compactly coherent spaces. 
Since we will be considering two different topologies on the same type, we need to define a type synonym in order for Lean to recognize which topology we are talking about. 
We set \lstinline|def CompactCoherentification (X : Type*) := X| and abbreviate it to \lstinline|k X|. 
This means that \lstinline|X| and \lstinline|k X| are definitionally equivalent but this equality should not be abused. \todo{Write why above? Cite something?}
Instead one should move between these topologies using a bijection: 
\begin{lstlisting}[frame=single]
protected def mk (X : Type*) : X ≃ CompactCoherentification X := Equiv.refl _
\end{lstlisting}
Now, we can provide a topology on \lstinline|k X| in the following way: 

\begin{lstlisting}[frame=single]
instance instTopologicalSpace : TopologicalSpace (k X) :=
  .coinduced (.mk X) (⨆ (K : Set X) (_ : IsCompact K), .coinduced (↑) (inferInstanceAs <| TopologicalSpace K))
\end{lstlisting}

Where we set our new topology two be coinduced by the disjoint union topology of all the compact subsets \lstinline|K| of \lstinline|X|. \todo[comment]{I don't actually understand myself what is going on in this definition.}
We prove that this definition implies that a set in our new topology is open if and only if its intersection with every compact set is open in the subspace topology of that compact set induced by the original topology. 
We show the equivalent statement for closed sets, prove that the new topology is finer than the original one and formalize conditions under which maps to, from and between compact coherentifications are continuous. 
Lastly, we show that the compact coherentification does indeed make an arbitrary topological space into a compactly coherent space. 

\subsection{Constructing the product}

\todo[plan]{Give a fairly detailed mathematical proof of the product here (a little less detailed than in my thesis).}

We want to use this subsection to not only discuss the implementations but also give a fairly detailed adaptation of the proof in \cite{Hatcher2002} to relative CW complexes. 
For the rest of the section let $C$ be a CW complex with base $D$ and $E$ be a CW complex with base $F$.
The respective families of characteristic maps are $(Q_i^n \colon D^n \to C)_{n \in \bN, i \in I_n}$ and $(P_j^m \colon D^m \to E)_{m \in \bN, j \in J_n}$. 
We will write the cells of $C$ as $\openCell{n}{i}$ and the cells of $E$ as $\openCellf{m}{j}$.
The theorem we aim to prove is the following:

\begin{thm}\label{thm:product}
  There is a CW structure on $\kif (C \times E)$ with base $(D \times E) \cup (C \times F)$ and characteristic maps $(Q_i^n \times P_j^m \colon D^n \times D^m \to \kif(C \times E))_{n,m \in \bN,i \in I_n,j \in J_m}$.
  The indexing sets $(K_l)_{l \in \bN}$ are given by $K_l = \bigcup_{n + m = l}I_n \times J_m$ for every $l \in \bN$ and the cells are therefore of the form $\openCell{n}{i} \times \openCellf{m}{j}$ for $n, m \in \bN$, $i \in I_n$ and $j \in J_m$.
\end{thm}

In Lean, we make the following assumptions for the entire section: 

\begin{lstlisting}[frame=single]
variable {X : Type*} {Y : Type*} [TopologicalSpace X] [TopologicalSpace Y] 
  {C D : Set X} {E F : Set Y}
\end{lstlisting}

The above theorem then translates to

\begin{lstlisting}[frame=single]
instance RelCWComplex.ProductCompactCoherentification [RelCWComplex C D] 
    [RelCWComplex E F] : 
    RelCWComplex (X := k (X × Y)) (C ×ˢ E) (D ×ˢ E ∪ C ×ˢ F) := 
  sorry
\end{lstlisting}

where \lstinline|sorry| is a way to state that we do not know the proof yet. 
We define the indexing type of the cells of the product as 

\begin{lstlisting}[frame=single]
structure RelCWComplex.prodCell (C : Set X) {D : Set X} (E : Set Y) {F : Set Y} 
    [RelCWComplex C D] [RelCWComplex E F] (n : ℕ) where
  m : ℕ
  l : ℕ
  hml : m + l = n
  j : cell C m
  k : cell E l
\end{lstlisting}

and the characteristic maps as

\begin{lstlisting}[frame=single]
def RelCWComplex.prodMap [RelCWComplex C D] [RelCWComplex E F] {n : ℕ} 
    (e : prodCell C E n) : PartialEquiv (Fin n → ℝ) (X × Y) :=
  (prodIsometryEquiv e.hml).transPartialEquiv
  (PartialEquiv.prod (map e.m e.j) (map e.l e.k))
\end{lstlisting}

where \lstinline|PartialEquiv.prod (map e.m e.j) (map e.l e.k)| is the product of the two relevant characteristic maps of \lstinline|C| and \lstinline|D| and \lstinline|prodIsometryEquiv| is the natural isometric isomorphism between \lstinline|Fin n → ℝ| and \lstinline|(Fin m → ℝ) × (Fin l → ℝ)| when \lstinline|m + l = n|.

We will focus on the two most important properties: weak topology and closure finiteness. 
First let us show that in the compact coherentification, the weak topology and the product topology agree. 

\begin{lem}\label{lem:weaktopologyproduct}
    $\kif (C \times E)$ has weak topology,
    i.e.\ $A \subseteq \kif (C \times E)$ is closed if $((D \times E) \cup (C \times F)) \cap A$ and $\closedCell{n}{i} \times \closedCellf{m}{j} \cap A$ are closed for all $n, m \in \bN$, $i \in I_n$ and $j \in J_m$. \todo[comment]{This proof should probably be a lot shorter}
\end{lem}
\begin{proof}
  Let $A \subseteq C \times E$ be a set such that $((D \times E) \cup (C \times F)) \cap A$ and $\closedCell{n}{i} \times \closedCellf{m}{j} \cap A$ are closed for all $n, m \in \bN$, $i \in I_n$ and $j \in J_m$. 
  We need to show that $A$ is closed in $\kif (C \times E)$.
  We know that the compact coherentification is a compactly coherent space and that $A$ is closed if for every compact set $K \subseteq \kif (C \times E)$, $A \cap K$ is closed in $K$.
    Take such a compact set $K$.
    The projections $\pr{1}{K}$ and $\pr{2}{K}$ are compact as images of a compact set. 
    Thus, there are finite sets $G \subseteq \{e_i^n \mid n \in \bN, i \in I_n \}$ and $H \subseteq \{f_j^m \mid m \in \bN, j \in J_m \}$ such that $\pr{1}{K} \subseteq D \cup \bigcup_{e \in G} e$ and $\pr{2}{K} \subseteq E \cup \bigcup_{f \in H} f$.
    Then,
    \begin{equation*}
      K \subseteq \pr{1}{K} \times \pr{2}{K} \subseteq \left( D \cup \bigcup_{e \in G} e \right) \times \left( E \cup \bigcup_{f \in H} f\right).
    \end{equation*}
    \begin{claim}
      It suffices to show that $A \cap \left( D \cup \bigcup_{e \in G} e \right) \times \left( E \cup \bigcup_{f \in H} f\right)$ is closed.
      \begin{proof}
        Indeed, since 
        \begin{equation*}
          A \cap K \subseteq A \cap \left( D \cup \bigcup_{e \in G} e \right) \times \left( E \cup \bigcup_{f \in H} f\right),
        \end{equation*}
        we get
        \begin{equation*}
          A \cap K = K \cap \left(A \cap \left( D \cup \bigcup_{e \in G} e \right) \times \left( E \cup \bigcup_{f \in H} f\right) \right)
        \end{equation*}
        which is closed as the intersection of two closed sets and therefore in particular closed in $K$. 
      \end{proof} 
    \end{claim}
    Now observe that 
    \begin{align*}
      A \cap &\left( D \cup \bigcup_{e \in G} e \right) \times \left( E \cup \bigcup_{f \in H} f\right) \\
      &= \left( A \cap \left( D \times F \cup \bigcup_{e \in G} e \times F \cup \bigcup_{f \in H} D \times f \right) \right) \cup \left( A \cap \bigcup_{e \in G} \bigcup_{f \in G} e \times f\right).
    \end{align*}
    \begin{claim}
      $A \cap \left( D \times F \cup \bigcup_{e \in G} e \times F \cup \bigcup_{f \in H} D \times f \right)$ is closed. 
      \begin{proof}
        We see that 
        \begin{equation*}
        D \times F \cup \bigcup_{e \in G} e \times F \cup \bigcup_{f \in H} D \times f \subseteq (D \times E) \cup (C \times F)
        \end{equation*}
        and thus 
        \begin{align*}
          A \cap &\left( D \times F \cup \bigcup_{e \in G} e \times F \cup \bigcup_{f \in H} D \times f \right) \\
          &= (A \cap ((D \times E) \cup (C \times F))) \cap \left( D \times F \cup \bigcup_{e \in G} e \times F \cup \bigcup_{f \in H} D \times f \right)
        \end{align*}
        which is closed as the intersection of two closed sets.
      \end{proof}
    \end{claim}
    \begin{claim}
      $A \cap \bigcup_{e \in G} \bigcup_{f \in G} e \times f$ is closed.
      \begin{proof}
        Since
        \begin{equation*}
          A \cap \bigcup_{e \in G} \bigcup_{f \in G} e \times f = \bigcup_{e \in G} \bigcup_{f \in G} A \cap e \times f,
        \end{equation*}
        this set is closed as a finite union of closed sets. 
      \end{proof}
    \end{claim}
    We see that the last two claims yield the desired result. 
\end{proof}

Now let us move onto closure finiteness: 

\begin{lem}
  $\kif (C \times E)$ has closure finiteness, i.e.\ each frontier of a cell is contained in the union of $(D \times E) \cup (C \times F)$ with a finite union of closed cells of a lower dimension.
\end{lem}
\begin{proof}
  We consider a cell $\openCell{n}{i} \times \openCellf{m}{j}$ for $n, m \in \bN$, $i \in I_n$ and $j \in J_m$. 
  First observe that its frontier is $\cellFrontier{n}{i} \times \closedCellf{m}{j} \cup \closedCell{n}{i} \times \cellFrontierf{m}{j}$.
  We can verify the claim separately for $\cellFrontier{n}{i} \times \closedCellf{m}{j}$ and $\closedCell{n}{i} \times \cellFrontierf{m}{j}$. 
  As both proofs are analogous, we will only do the former.
  Since $C$ fulfils closure finiteness, there is a finite set $G$ of cells of $C$ of dimension less than $n$ such that $\cellFrontier{n}{i} \subset D \cup \bigcup_{e \in G}\closure{e}$. 
  This gives us 
  \begin{equation*}
    \cellFrontier{n}{i} \times \closedCellf{m}{j} \subseteq \left( D \cup \bigcup_{e \in G}\closure{e} \right) \times \closedCellf{m}{j} = D \times \closedCellf{m}{j} \cup \bigcup_{e \in G} \closure{e} \times \closedCellf{m}{j} \subseteq ((D \times E) \cup (C \times F)) \cup \bigcup_{e \in G} \closure{e} \times \closedCellf{m}{j},
  \end{equation*}
  which is the union of the base with a finite union of closed cells of $\kif (X \times Y)$ of dimension less than $n + m$.
\end{proof}

The rest of the proof of Theorem \ref{thm:product} is straightforward. 

We can then derive an instance for absolute CW complexes 

\begin{lstlisting}[frame=single]
instance CWComplex.ProductCompactCoherentification [CWComplex C] [CWComplex E] :
    CWComplex (X := k (X × Y)) (C ×ˢ E) :=
  (RelCWComplex.ofEq (X := k (X × Y)) (C ×ˢ E) (∅ ×ˢ E ∪ C ×ˢ ∅) rfl 
    (by simp)).toCWComplex
\end{lstlisting}

and prove finiteness properties about these instances. 