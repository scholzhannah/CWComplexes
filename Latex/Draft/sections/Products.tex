\section{Products}\label{sec:products}

In general, the product of two CW complexes is not necessarily a CW complex because the weak topology of the CW complex might not match the product topology. \todo{mention term ``weak topology'' already somewhere in definition section (or even introduction?)}
A counterexample was first provided by \citeauthor{Dowker1952} in \cite{Dowker1952}.

In order to achieve the correct topology on the product space we need to turn it into a compactly generated space which we will discuss in the next subsection.

\subsection{Compactly generated spaces}

The name ``compactly generated space'' is used for different notions in the literature.
Firstly, it can refer to a space with a topology that is coherent with its compact subsets, i.e.\ a set is closed iff its intersection with every compact subset is closed in that subset. 
Secondly, it can refer to a space with a topology determined by continuous maps from compact Hausdorff spaces, i.e.\ a set is closed iff its preimage under every continuous map from a compact Hausdorff map is closed. 
Thirdly, it can refer to a space with a topology coherent with its compact Hausdorff subspaces, i.e.\ a set is closed iff its intersection with every compact Hausdorff subspace is closed in that subspace. \todo[comment]{Cite wikipedia?}

While these three notions agree for Hausdorff spaces, in the general case, the first is the weakest and the third the strongest. 
When starting this formalization, the second version was already in \mathlib as \lstinline|CompactlyGeneratedSpace|; the two other versions had not been formalized. 
We intended to follow the construction of the product presented in \cite{Hatcher2002} which uses the first version of compactly generated spaces. 
Since we assume our ambient space to be Hausdorff, we could have just translated the proof to use the version already in \mathlib. 
Instead, we decided to formalize the first version and named it \lstinline|CompactlyCoherentSpace|.\todo[comment]{Credit person that came up with name?}

In \mathlib the definition is the following: 

\begin{lstlisting}[frame=single]
class CompactlyCoherentSpace (X : Type*) [TopologicalSpace X] : Prop where
  isCoherentWith : IsCoherentWith (X := X) {K | IsCompact K}
\end{lstlisting}

which uses the already pre-existing structure \lstinline|IsCoherentWith| that is defined as: 

\begin{lstlisting}[frame=single]
structure IsCoherentWith (S : Set (Set X)) : Prop where
  isOpen_of_forall_induced (u : Set X) : 
    (∀ s ∈ S, IsOpen ((↑) ⁻¹' u : Set s)) → IsOpen u
\end{lstlisting}\todo[comment]{Credit author?}
\todo[plan]{Explain this definition in more detail. Explain the arrow. Explain equivalence of openness and closedness condition (use open above?)}

\todo[plan]{Talk about results formalized}

\subsection{Product of CW complexes}
\todo{What does one name this subsection?}

\todo[plan]{Give a fairly detailed mathematical proof of the product here (a little less detailed than in my thesis).}
