\section{Implicit arguments and typeclass inference}
\label{sec:implicitandtypeclass}

A crucial factor that makes Lean more comfortable to use and makes the formalisation process feel closer to doing mathematics on paper is its use of \emph{implicit arguments} and \emph{typeclass inference}. 
We will explain both of these concepts in this section.

First let us discuss implicit arguments based on \cite{Avigad2024}. 
One way that we could define continuity in Lean is the following\cprotect\footnote{The code in this section will run if \lstinline{import Mathlib.Topology.MetricSpace.Basic} is written at the top of the file.}:

\begin{lstlisting}
structure Continuous' (X Y : Type*) (t : TopologicalSpace X) 
    (s : TopologicalSpace Y) (f : X → Y) : Prop where
  isOpen_preimage : ∀ s, IsOpen s → IsOpen (f ⁻¹' s)
\end{lstlisting}

Where a structure is a construct that can bundle both data and properties after the keyword \lstinline{where}. 
This structure has no data and one property which is named \lstinline{isOpen_preimage}. 
\lstinline{f ⁻¹' s} denotes the preimage of \lstinline{s} under \lstinline{f}.
But now if we are given two types \lstinline{X} and \lstinline{Y} with topologies \lstinline{s} and \lstinline{t} respectively and a map \lstinline{f : X → Y}, the statement that the map \lstinline{f} is continuous would be expressed in the following way:

\begin{lstlisting}
example (X Y : Type*) (t : TopologicalSpace X) (s : TopologicalSpace Y) 
  (f : X → Y) : Continuous' X Y t s f := ⋯
\end{lstlisting}

where everything before the colon is the context we described above and after the colon equal you could write a proof.

One thing that we can notice is that the types \lstinline{X} and \lstinline{Y} are contained in the definition of \lstinline{f} which means that Lean should be able to find that information itself. 
To tell Lean to do that you can replace the variables by underscores: 

\begin{lstlisting}
  example (X Y : Type*) (t : TopologicalSpace X) (s : TopologicalSpace Y) 
  (f : X → Y) : Continuous' _ _ t s f := ⋯
\end{lstlisting}

These two arguments are always clear from the context in this way. 
We therefore want to specify in the definition that they should not be given explicitly but instead inferred by the system.
We use curly brackets to do this: 

\begin{lstlisting}
structure Continuous'' {X Y : Type*} (t : TopologicalSpace X) 
    (s : TopologicalSpace Y) (f : X → Y) : Prop where
  isOpen_preimage : ∀ s, IsOpen s → IsOpen (f ⁻¹' s)
\end{lstlisting}

which enables us to write continuity like this: 

\begin{lstlisting}
example (X Y : Type*) (t : TopologicalSpace X) (s : TopologicalSpace Y) 
  (f : X → Y) : Continuous'' t s f := ⋯
\end{lstlisting}

This is already a lot shorter than what we had above but there is still room for improvement as on paper we would probably just write "$f$ is continuous" since in most contexts $X$ and $Y$ will only have one specified topology each that can be inferred by the reader.
The same thing is also true in Lean and we can achieve this by typeclass inference.
Typeclasses were first invented by \Citeauthor{Wadler1989} in \cite{Wadler1989} to be used in the programming language Haskell. 
They are a way to overload operations for various different types. 
For example you might want to write code that works for all types that have a topology. 
In Lean this is possible by just stating that you input type \lstinline{X} is part of the typeclass \lstinline{TopologicalSpace}. 
You can specify that something ia a typeclass with the keyword \lstinline{class}. 
The definition of the typeclass of topological spaces in mathlib looks like this

\begin{lstlisting}
class TopologicalSpace (X : Type*) where
  protected IsOpen : Set X → Prop
  protected isOpen_univ : IsOpen univ
  protected isOpen_inter : ∀ s t, IsOpen s → IsOpen t → IsOpen (s ∩ t)
  protected isOpen_sUnion : ∀ s, (∀ t ∈ s, IsOpen t) → IsOpen (⋃₀ s)
\end{lstlisting}

Let us first explain what this code means: 
The keyword \lstinline{protected} means that these properties should not be accessed directly because there are lemmas that should be used instead. 
\lstinline{Set X} is the type that consists of all sets of elements of \lstinline{X}.
Thus the line \lstinline{protected IsOpen : Set X → Prop} expresses that \lstinline{IsOpen} is a property that can be assigned to a set in \lstinline{X}.
The rest of the lines discuss the properties of a topology.
\lstinline{univ} is the set that is composed of all elements of \lstinline{X} and \lstinline{⋃₀ s} is the union over the set \lstinline{s}. 
All of these explanations are not actually relevant to typeclasses, they are just for your understanding of the above code. 

Typeclasses are also expected to be inferred automatically. 
Local instances of these typeclasses can written with square brackets which tells Lean to infer these automatically.

We can now look at the version of continuity that is almost identical to that of mathlib: 

\begin{lstlisting}
structure Continuous {X Y : Type*} [t : TopologicalSpace X]
    [s : TopologicalSpace Y] (f : X → Y) : Prop where
  isOpen_preimage : ∀ s, IsOpen s → IsOpen (f ⁻¹' s)
\end{lstlisting}

which enables us to write that \lstinline{f} is continuous in the context explained above as follows:

\begin{lstlisting}
example (X Y : Type*) (t : TopologicalSpace X) (s : TopologicalSpace Y) 
    (f : X → Y) : Continuous f := ⋯
\end{lstlisting}

When you define a class you can then define instances of that class to be inferred whenever you talk you talk about a type with that instance. 
Mathlib defines the discrete topology on $\bZ$ as an instance: 

\begin{lstlisting}
instance : TopologicalSpace ℤ := ⊥
\end{lstlisting}

where \lstinline{⊥} is the smallest element in the order that can be defined on the topologies of a space, i.e. the finest topology which is the discrete topology. 
That makes it so that for any map \lstinline{f : ℤ → ℤ} we can just write the following

\begin{lstlisting}
example (f : ℤ → ℤ) : Continuous f := ⋯
\end{lstlisting} 

and Lean automatically knows which topology we are talking about. 
We can additionally say that an instance implies another instance. 
If you have types \lstinline{X} and \lstinline{Y} which both have topologies defined on them this instance in mathlib gives you a topology on the product: 

\begin{lstlisting}
instance instTopologicalSpaceProd {X Y : Type*} [t₁ : TopologicalSpace X] 
  [t₂ : TopologicalSpace Y] : TopologicalSpace (X × Y) := ⋯
\end{lstlisting}

which enables you to write the following

\begin{lstlisting}
example {X Y Z : Type*} [TopologicalSpace X] [TopologicalSpace Y]
  [TopologicalSpace Z] (f : X × Y → Z) : Continuous f := ⋯
\end{lstlisting}

This also works across different typeclasses. 
We can write 

\begin{lstlisting}
example {X : Type*} [MetricSpace X] (f : X → X) : Continuous f := ⋯
\end{lstlisting}

which works because Lean knows that a metric space is by definition a pseudometric space which is a uniform space which is by definition a topological space.