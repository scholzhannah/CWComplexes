\section{The type theory of Lean}
\label{sec:typetheory}

\emph{Type theory} was first proposed by \Citeauthor{Russell1908} in \citeyear{Russell1908} \cite{Russell1908} as a way to axiomatise mathematics and resolve the paradoxes (most famously Russell's paradox) that were discussed at the time. 
While type theory has lost its relevance as a foundation of mathematics to set theory it has since been studied in both mathematics and computer science. 
It was first used in formal mathematics in 1967 in the formal language \emph{AUTOMATH}. 
More about the history of type theory can be found in \cite{Kamareddine2004}. 
Discussions of type theory in mathematics and especially its connections to homotopy theory forming the new area of \emph{homotopy type theory} can be found in \cite{hottbook}.
We will now focus on the type theory as used in Lean.
Its type theory along with the type theory of other proof assistants such as \emph{Coq} are based on \emph{constructive type theory} developed by Per Martin-Löf which makes use of dependent types \cite{Martin-Löf1984}.
A detailed account of Lean's type theory can be found in \cite{Carneiro2019}. 
The following short discussion is based on \cite{Avigad2024}. 

Lean uses what is called a \emph{dependent type theory}.
In type theory every object has a type. 
A type can for example be the natural numbers or propositions, which we write in Lean as \lstinline{Nat} or \lstinline{ℕ} and \lstinline{Prop} respectively.
To assert that \lstinline{n} is a natural number or that \lstinline{p} is a proposition we write \lstinline{n : ℕ} and \lstinline{p : Prop}. 
Proofs of a proposition \lstinline{p} also form a type written as \lstinline{p}.
If you want to say that \lstinline{hp} is a proof of \lstinline{p} then you can simply write \lstinline{hp : p}.
Something to note about proofs in Lean is that, contrary to other type theories, the type theory of Lean has \emph{proof irrelevance} which means that two proofs of a proposition \lstinline{p} are by definition assumed to be the same.

Even types themselves have types. 
In Lean the type of natural numbers \lstinline{ℕ} has the type \lstinline{Type}. 
The type of propositions \lstinline{Prop} is also of type \lstinline{Type}.
As to not run into a paradox called Girard's paradox there is a hierarchy of types \cite{Coquand1986}. 
The type of \lstinline{Type} is \lstinline{Type 1}, the type of \lstinline{Type 1} is \lstinline{Type 2} and so on.
These are called type-universes. 
The notation \lstinline{α : Type*} is a way of stating that \lstinline{α} is a type in an arbitrary universe.

There are a few ways to construct new types from existing ones. 
Some of them are very similar to constructions on sets such as the cartesian product of types written as \lstinline{α × β} or the type of functions from \lstinline{α} to \lstinline{β} written as \lstinline{α → β} where \lstinline{α} and \lstinline{β} are types.
Elements of \lstinline{α × β} can be written as \lstinline{(a, b)} for every \lstinline{a : α} and \lstinline{b : β}.
Elements of \lstinline{α → β} can be written as \lstinline{fun a ↦ s a} for some \lstinline{s : α → β}. 
Since these are quite self-explanatory, we will not go into more detail.
We will now mainly discuss constructions that do not fulfil this criterium. 
A first example is the sum type of two types \lstinline{α} and \lstinline{β} written as \lstinline{α ⊕ β} which is the equivalent to a disjoint union of sets. 
Elements of this type are of the form \lstinline{Sum.inl a} for \lstinline{a : α} or \lstinline{Sum.inr b} for \lstinline{b : β}.
When given an \lstinline{x : α ⊕ β} we can use the construction 

\begin{lstlisting}
match x with 
| Sum.inl a => ⋯ 
| Sum.inr b => ⋯
\end{lstlisting}

to write two different definitions or proofs depending on whether \lstinline{x} originates from an element of \lstinline{α} or \lstinline{β}. 
In this code snippet the names \lstinline{a} and \lstinline{b} are arbitrary.

The next two examples explain why this type theory is a dependent type theory:
If we have a type \lstinline{α} and for every \lstinline{(a : α)} a type \lstinline{β a} (i.e. \lstinline{β} is a function assigning a type to every \lstinline{a : α}) then we can construct the \emph{pi type} or dependent function type written as \lstinline{(a : α) → β a} or \lstinline{Π (a : α), β a}. 
We can construct an element of this type by writing \lstinline{fun a ↦ s a} for some \lstinline{s : (a : α) → β a}. 
Here is an example: 
Assume that you want a function that for every pair in a cartesian product \lstinline{α × α} for any type \lstinline{α} returns the first element. 
Then this would be a function that depends on \lstinline{α} and whose type is therefore the dependent function type \lstinline{(α : Type*) → α × α → α}.

The dependent version of the cartesian product is called a \emph{sigma type} and can be written as \lstinline{(a : α) × β a} or \lstinline{Σ a : α, β a} for \lstinline{α} and \lstinline{β} the same way as above. 
An element of the sigma type can be written as \lstinline{⟨a, b⟩} for \lstinline{a : α} and \lstinline{b : β a}. 
When given an  element of the sigma type \lstinline{x : Σ a : α, β a} one can write \lstinline{obtain ⟨a, b⟩ := x} to deconstruct \lstinline{x}.