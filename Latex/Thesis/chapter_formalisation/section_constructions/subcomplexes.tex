\subsection{Subcomplexes}

With the assumptions for the rest of the section

\begin{lstlisting}
variable {X : Type*} [t : TopologicalSpace X] [T2Space X] {C : Set X} 
  [CWComplex C]
\end{lstlisting}

this is our definition of a subcomplex:

\begin{lstlisting}
class Subcomplex (C : Set X) [CWComplex C] (E : Set X) where
  I : Π n, Set (cell C n)
  closed : IsClosed E
  union : ⋃ (n : ℕ) (j : I n), openCell (C := C) n j = E
\end{lstlisting}

The \lstinline{(C := C)} is used to specify which CW-complex we are talking about. 
We need to use this notation because we made \lstinline{C} and implicit variable in the definition of \lstinline{openCell}.

We chose this to be a class again so that when you want to talk about the CW-structure of a subcomplex you don't need to explicitly mention the subcomplex structure and that every subcomplex is a CW-complex.

But we still have one issue: 
We would like the instance 

\begin{lstlisting}
instance doesnotcompile (E : Set X) [subcomplex : Subcomplex C E] : 
  CWComplex E := ⋯ 
\end{lstlisting}

which unfortunately does not work as Lean has no way to infer \lstinline{C} from \lstinline{CWComplex E} and therefore does not know that it should be looking for the instance \lstinline{[subcomplex : Subcomplex C E]}. 

To remedy this we can define some notation that includes the variable \lstinline{C}: 

\begin{lstlisting}
def Sub (E : Set X) (C : Set X) [CWComplex C] [Subcomplex C E] : Set X := E

scoped infixr:35 " ⇂ "  => Sub
\end{lstlisting}

The first line defines a synonym for \lstinline{E} that in its context includes that \lstinline{E} is a subcomplex of the CW-complex \lstinline{C}. 
The second line then defines the actual notation. 

We can now use this to make the instance from above work: 

\begin{lstlisting}
instance CWComplex_subcomplex (E : Set X) [subcomplex : Subcomplex C E] : CWComplex (E ⇂ C) := ⋯
\end{lstlisting}

With that we can prove the lemmas from Section \ref{sec:mathsubcomplex}. 
Here are two examples.
They correspond to statements \ref{lem:cellinfinitesubcomplex} and \ref{cor:compactinfinitesubcomplex}.

\begin{lstlisting}
lemma cell_mem_finite_subcomplex (n : ℕ) (i : cell C n) :
    ∃ (E : Set X) (subE : Subcomplex C E), Finite (E ⇂ C) ∧ i ∈ subE.I n := ⋯

lemma compact_subset_finite_subcomplex {B : Set X} (compact : IsCompact B) :
  ∃ (E : Set X) (sub : Subcomplex C E), CWComplex.Finite (E ⇂ C) ∧ B ∩ C ⊆ E := ⋯
\end{lstlisting}