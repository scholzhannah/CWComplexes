\subsection{Miscellaneous constructions}

Formalising the CW-complex structure on the $n$-skeleton, the disjoint union and the image of CW-complex under a homeomorphism is relatively straightforward. 
In the first two we take advantage of the relative approach of looking at subspaces as CW-complexes as this helps us avoid having to deal with the subspace and disjoint topology. 

The following code snippet includes the statements and assumptions:

\begin{lstlisting}
def CWComplex_of_Homeomorph.{u} {X Y : Type  u} [TopologicalSpace X] [TopologicalSpace Y] (C : Set X) (D : Set Y) [CWComplex C] 
    (f : X ≃ₜ Y) (imf : f '' C = D) : CWComplex D := ⋯
\end{lstlisting}
\href{https://github.com/scholzhannah/CWComplexes/blob/7be4872a05b534011cc969eb5b80a4b7f0bf57e2/CWcomplexes/Constructions.lean#L140-L182}{\faExternalLink}

\begin{lstlisting}
variable {X : Type*} [t : TopologicalSpace X] [T2Space X] (C : Set X) [CWComplex C]

instance CWComplex_level (n : ℕ∞) : CWComplex (level C n) := ⋯
\end{lstlisting}
\href{https://github.com/scholzhannah/CWComplexes/blob/7be4872a05b534011cc969eb5b80a4b7f0bf57e2/CWcomplexes/Constructions.lean#L66-L67}{\faExternalLink}

\begin{lstlisting}
variable {D : Set X} [CWComplex D] 

def CWComplex_disjointUnion (disjoint : Disjoint C D) : 
    CWComplex (C ∪ D) := ⋯
\end{lstlisting}
\href{https://github.com/scholzhannah/CWComplexes/blob/7be4872a05b534011cc969eb5b80a4b7f0bf57e2/CWcomplexes/Constructions.lean#L71-L136}{\faExternalLink}

Where \lstinline{f : X ≃ₜ Y} is the statement that \lstinline{f} is a homeomorphism
\href{https://github.com/leanprover-community/mathlib4/blob/ed125a4216d18273cb1b96d4c846d32b85d74faf/Mathlib/Topology/Homeomorph.lean#L37-L43}{\faExternalLink}. 
Note that the first and last constructions are not instances.
This is because the typeclass inference has no way to know that it should look for the hypotheses \lstinline{(f : X ≃ₜ Y) (imf : f '' C = D)} and \lstinline{(disjoint : Disjoint C D)} as this information is not contained in the statement. 
So labeling these as instances would not be helpful. 
But in \lstinline{CWComplex_level} all the necessary information is contained in the statement \lstinline{CWComplex (level C n)}, which means that this will work as an instance.