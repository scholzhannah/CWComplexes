\subsection{Product of CW-complexes}

Let us lastly take a look at the product. 

We define k-spaces as follows: 
\href{https://github.com/scholzhannah/CWComplexes/blob/7be4872a05b534011cc969eb5b80a4b7f0bf57e2/CWcomplexes/KSpace.lean#L33-L39}{\faExternalLink}

\begin{lstlisting}
class KSpace  (X : Type*) [TopologicalSpace X] where
  isOpen_iff A : IsOpen A ↔
  ∀ (B : Set X), IsCompact B → ∃ (C : Set X), IsOpen C ∧ A ∩ B = C ∩ B
\end{lstlisting}

When defining the k-ification, we need to be a little careful. 
We want it to be an instance derived from another instance defined on the same type. 
To tell the system what instance we are referring to, we define a type synonym and then the k-ification: 
\href{https://github.com/scholzhannah/CWComplexes/blob/7be4872a05b534011cc969eb5b80a4b7f0bf57e2/CWcomplexes/KSpace.lean#L119-L146}{\faExternalLink}

\begin{lstlisting}
def kification (X : Type*) := X

instance instkification {X : Type*} [t : TopologicalSpace X] : TopologicalSpace (kification X) := ⋯
\end{lstlisting}

Sometimes it is convenient to use maps that go to or from the k-ification, we define them as bijections:
\href{https://github.com/scholzhannah/CWComplexes/blob/7be4872a05b534011cc969eb5b80a4b7f0bf57e2/CWcomplexes/KSpace.lean#L164-L170}{\faExternalLink}

\begin{lstlisting}
def tokification (X : Type*) : X ≃ kification X :=
  ⟨fun x ↦ x, fun x ↦ x, fun _ ↦ rfl, fun _ ↦ rfl⟩
  
def fromkification (X : Type*) : kification X ≃ X :=
  ⟨fun x ↦ x, fun x ↦ x, fun _ ↦ rfl, fun _ ↦ rfl⟩
\end{lstlisting}

Here are the statements of some of the lemmas in Section \ref{sec:kspace}. 
They correspond to the statements \ref{lem:weaklylocallycompactiskspace}
\href{https://github.com/scholzhannah/CWComplexes/blob/7be4872a05b534011cc969eb5b80a4b7f0bf57e2/CWcomplexes/KSpace.lean#L69-L86}{\faExternalLink},
\ref{lem:sequentialiskspace}
\href{https://github.com/scholzhannah/CWComplexes/blob/7be4872a05b534011cc969eb5b80a4b7f0bf57e2/CWcomplexes/KSpace.lean#L88-L114}{\faExternalLink},
\ref{lem:compactiffcompact}
\href{https://github.com/scholzhannah/CWComplexes/blob/7be4872a05b534011cc969eb5b80a4b7f0bf57e2/CWcomplexes/KSpace.lean#L196-L236}{\faExternalLink},
\ref{lem:kificationiskspace}
\href{https://github.com/scholzhannah/CWComplexes/blob/7be4872a05b534011cc969eb5b80a4b7f0bf57e2/CWcomplexes/KSpace.lean#L245-L258}{\faExternalLink}
and \ref{lem:continuousofcontinuousoncompact}
\href{https://github.com/scholzhannah/CWComplexes/blob/7be4872a05b534011cc969eb5b80a4b7f0bf57e2/CWcomplexes/KSpace.lean#L303-L316}{\faExternalLink}.

\begin{lstlisting}
instance kspace_of_WeaklyLocallyCompactSpace {X : Type*}[TopologicalSpace X]
    [WeaklyLocallyCompactSpace X] : 
  KSpace X := ⋯

instance kspace_of_SequentialSpace {X : Type*} [TopologicalSpace X] [SequentialSpace X]: KSpace X := ⋯

lemma isCompact_iff_isCompact_tokification_image {X : Type*} [TopologicalSpace X] (C : Set X) :
  IsCompact C ↔ IsCompact (tokification X '' C) := ⋯

instance kspace_kification {X : Type*} [TopologicalSpace X] : 
  KSpace (kification X) := ⋯

lemma continuous_kification_of_continuousOn_compact {X Y : Type*} 
    [tX : TopologicalSpace X] [tY : TopologicalSpace Y] (f : X → Y) 
    (conton : ∀ (C : Set X), IsCompact C → ContinuousOn f C) :
  Continuous (X := kification X) (Y := kification Y) f := ⋯
\end{lstlisting}

When proving statements about different topologies on the same space, formalising in Lean can be very helpful to keep track of what topology needs to be considered when. 

With that we can move on to the product. 
The assumptions for the rest of the section are:

\begin{lstlisting}
variable {X : Type*} {Y : Type*} [t1 : TopologicalSpace X] 
  [t2 : TopologicalSpace Y] [T2Space X] [T2Space Y] {C : Set X} {D : Set Y} 
  [CWComplex C] [CWComplex D]
\end{lstlisting}

Since the proof of Theorem \ref{thm:productcw} is both long and technical it is useful to separate out some definitions and lemmas. 

We first define the indexing sets in the product:
\href{https://github.com/scholzhannah/CWComplexes/blob/7be4872a05b534011cc969eb5b80a4b7f0bf57e2/CWcomplexes/Product.lean#L29-L31}{\faExternalLink}

\begin{lstlisting}
def prodcell (C : Set X) (D : Set Y) [CWComplex C] [CWComplex D] (n : ℕ) :=
    (Σ' (m : ℕ) (l : ℕ) (hml : m + l = n), cell C m × cell D l)
\end{lstlisting}

where the \lstinline{Σ'} is a sigma type that allows us to add properties, in this case \lstinline{hml}. 

When discussing the product in Section \ref{sec:constructproduct}, we always identified $\bR^{m + n}$ and $\bR^m \times \bR^n$. 
When formalising we need to make that identification explicit: 
\href{https://github.com/scholzhannah/CWComplexes/blob/7be4872a05b534011cc969eb5b80a4b7f0bf57e2/CWcomplexes/Product.lean#L33-L36}{\faExternalLink}

\begin{lstlisting}
def prodisometryequiv {n m l : ℕ}  (hmln : m + l = n) (j : cell C m) (k : cell D l) : (Fin n → ℝ) ≃ᵢ (Fin m → ℝ) × (Fin l → ℝ) := ⋯
\end{lstlisting}

where \lstinline{≃ᵢ} is a symbol used to denote a bijective isometry \href{https://github.com/leanprover-community/mathlib4/blob/ed125a4216d18273cb1b96d4c846d32b85d74faf/Mathlib/Topology/MetricSpace/Isometry.lean#L242-L246}{\faExternalLink}. 
Using this map, we can define the characteristic maps of the product: 
\href{https://github.com/scholzhannah/CWComplexes/blob/7be4872a05b534011cc969eb5b80a4b7f0bf57e2/CWcomplexes/Product.lean#L38-L41}{\faExternalLink}

\begin{lstlisting}
def prodmap {n m l : ℕ} (hmln : m + l = n) (j : cell C m) (k : cell D l) : PartialEquiv (Fin n → ℝ) (X × Y) := ⋯
\end{lstlisting}

After some lemmas about these maps such as for example 
\href{https://github.com/scholzhannah/CWComplexes/blob/7be4872a05b534011cc969eb5b80a4b7f0bf57e2/CWcomplexes/Product.lean#L65-L70}{\faExternalLink}

\begin{lstlisting}
lemma prodmap_image_sphere {n m l : ℕ} {hmln : m + l = n} {j : cell C m} {k : cell D l} :
    prodmap hmln j k '' sphere 0 1 = (cellFrontier m j) ×ˢ (closedCell l k) ∪
    (closedCell m j) ×ˢ (cellFrontier l k) := ⋯
\end{lstlisting}

we can move on to defining the product: 
\href{https://github.com/scholzhannah/CWComplexes/blob/7be4872a05b534011cc969eb5b80a4b7f0bf57e2/CWcomplexes/Product.lean#L226-L377}{\faExternalLink}

\begin{lstlisting}
instance CWComplex_product_kification : CWComplex (X := kification (X × Y)) 
  (C ×ˢ D) := ⋯
\end{lstlisting}

where \lstinline{×ˢ} is the product of sets.
Finally we define a version of this instance for when the product is already a k-space as to not unnecessarily apply the k-ification: 
\href{https://github.com/scholzhannah/CWComplexes/blob/7be4872a05b534011cc969eb5b80a4b7f0bf57e2/CWcomplexes/Product.lean#L81-L224}{\faExternalLink}

\begin{lstlisting}
instance CWComplex_product [KSpace (X × Y)] : CWComplex (C ×ˢ D) := ⋯
\end{lstlisting}