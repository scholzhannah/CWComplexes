\section{Definition and basic properties of a CW-complex}

The following is our definition of CW-complexes in Lean:

\begin{lstlisting}
class CWComplex.{u} {X : Type u} [TopologicalSpace X] (C : Set X) where
  cell (n : ℕ) : Type u
  map (n : ℕ) (i : cell n) : PartialEquiv (Fin n → ℝ) X
  source_eq (n : ℕ) (i : cell n) : (map n i).source = closedBall 0 1
  cont (n : ℕ) (i : cell n) : ContinuousOn (map n i) (closedBall 0 1)
  cont_symm (n : ℕ) (i : cell n) : ContinuousOn (map n i).symm (map n i).target
  pairwiseDisjoint' :
    (univ : Set (Σ n, cell n)).PairwiseDisjoint (fun ni ↦ map ni.1 ni.2 '' ball 0 1)
  mapsto (n : ℕ) (i : cell n) : ∃ I : Π m, Finset (cell m),
    MapsTo (map n i) (sphere 0 1) (⋃ (m < n) (j ∈ I m), map m j '' closedBall 0 1)
  closed' (A : Set X) (asubc : A ⊆ C) : IsClosed A ↔ ∀ n j, IsClosed (A ∩ map n j '' closedBall 0 1)
  union' : ⋃ (n : ℕ) (j : cell n), map n j '' closedBall 0 1 = C
\end{lstlisting}

The \lstinline|.{u}| is a way to fix a universe level so that our definition of a CW-complex does not depend on a number of different universe levels: The one of \lstinline{X} and the one of \lstinline{cell n} for every $n \in \bN$.
\lstinline{cell (n : ℕ)} represents the indexing set that we called $I_n$ in definition \ref{defi:CWComplex2}. \lstinline{map (n : ℕ) (i : cell n)} represent what we called $Q_i^n$ in that definition. 

\lstinline{Fin n} is the set containing $n$ natural numbers starting at 0. 
\lstinline{Fin n → ℝ} is one way to express $\bR^n$ in Lean. 
\lstinline{PartialEquiv} is a structure defined in mathlib as follows: 

\begin{lstlisting}
structure PartialEquiv (α : Type*) (β : Type*) where
  toFun : α → β
  invFun : β → α
  source : Set α
  target : Set β
  map_source' : ∀ ⦃x⦄, x ∈ source → toFun x ∈ target
  map_target' : ∀ ⦃x⦄, x ∈ target → invFun x ∈ source
  left_inv' : ∀ ⦃x⦄, x ∈ source → invFun (toFun x) = x
  right_inv' : ∀ ⦃x⦄, x ∈ target → toFun (invFun x) = x
\end{lstlisting}

It bundles two maps and two sets that get mapped two each other by the respective maps. 
Restricting the maps to these sets yields two maps that the inverse of each other. 
We use this instead of a similar construction called \lstinline{Equiv} for bijections to avoid explicitly having to deal with restrictions. 
The brackets \lstinline{⦃⦄} are similar to the curly brackets and are used here since \lstinline{x} can be inferred from the left sides of the implications.

The property \lstinline{source_eq} specifies the source of the \lstinline{PartialEquiv}. 
\lstinline{cont} and \lstinline{cont_symm} make the bijection into a homeomorphism giving us property (i) of definition \ref{defi:CWComplex2}.
The property \lstinline{pairwiseDisjoint'} corresponds to property (ii) of definition \ref{defi:CWComplex2}. 
We are adding the prime to its name because we will later see a lemma called \lstinline{pairwiseDisjoint} that we prefer to be used. 

\lstinline{mapsto} is the equivalent of property (iii) of the definition of a CW-complex. 
The \lstinline{Π} defines a dependent function type which we discussed in section \ref{sec:typetheory}.
\lstinline{Finset α} is the type of all finite sets in a type \lstinline{α}. 
It can be imagined as a set bundled with the information that it is finite (but note that the actual definitions of \lstinline{Finset α} and \lstinline{Set α} are quite different).
\lstinline{MapsTo} is defined as
\begin{lstlisting}
def MapsTo (f : α → β) (s : Set α) (t : Set β) : Prop := 
  ∀ ⦃x⦄, x ∈ s → f x ∈ t
\end{lstlisting}

and is relatively self-explanatory. 

\lstinline{closed'} represents property (iv) of definition \ref{defi:CWComplex2} and \lstinline{union'} represents property (v). 

\medskip

There are a few things to note about this formalisation of the definition. 
First of all it does not require $X$ to be a Hausdorff space. 
This is done so that when you define a CW-complex you can choose to first define the structure in this way and later show that it is a Haussdorff space to apply lemmas about CW-complexes of which almost all will require that $X$ is Hausdorff. 
Additionally we define what it means for a subspace $C$ of $X$ to be a CW-complex.
This is useful firstly to be able to work with a nicer topology: 
If you consider $S^1$ as a CW-complex and a subspace of $\bR$ you might find it easier to work with the topology on $\bR$ instead of the subspace topology. 
Secondly constructions such as attaching cells or taking disjoint unions of CW-complexes might be easier to work with if you are already working in the same overarching type.
This approach is inspired by \cite{Gonthier2013} where the authors notices that it was helpful to consider subsets of an ambient group to avoid having to work with different group operations and similar issues. 

One question that naturally arises is whether these changes to the definition preserve the notion of a CW-complex. 
Firstly note that if we choose $X$ and $C$ to be the same we recover definition \ref{defi:CWComplex2} exactly. 
If we choose $X$ and $C$ to be different we still guarantee that $C$ is a CW-complex: 

\begin{lem}
  Let $C$ be a CW-complex in a Hausdorff space $X$ as in the definition in the formalisation. 
  Then $C$ is a CW-complex as in definition \ref{defi:CWComplex2}. 
\end{lem}
\begin{proof}
  Properties (i), (ii), (iii) and (v) of definition \ref{defi:CWComplex2} are immediate. 
  Thus let us look at property (iv). 
  We assume that 
  \[A \subseteq C \text{ is closed in } X \iff \closedCell{n}{i} \cap A \text{ is closed in } X \text{ for all } n \in \bN \text{ and } i \in I_n\]
  and need to show that 
  \[A \subseteq C \text{ is closed in } C \iff \closedCell{n}{i} \cap A \text{ is closed in } C \text{ for all } n \in \bN \text{ and } i \in I_n.\]
  It is easy to see that the forward direction is true. 
  For the backwards direction take $A \subseteq C$ such that $A \cap C$ is closed for all 
\end{proof}