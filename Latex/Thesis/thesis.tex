\documentclass[paper=a4, fontsize=11pt, BCOR=13mm, DIV=13, headinclude, toc=index, toc=bibliography, english, twoside]{scrreprt}
% Die verwendete Dokumentenklasse ist scrreprt. Die verwendeten Optionen sind:
%
% paper=a4              Papier ist a4.
% fontsize=11pt         Schrifgröße ist 11.
% DIV=13                Das Papier wird in d viele Spalten und d' viele Zeilen eingeteilt. Die Werte werden aus DIV berechnet.
% BCOR=1cm              Definiert den Patz, der auf der Innenseite beim Binden verloren geht.
% headinclude           sorgt dafür, dass genug Platz für die Header vorhanden ist.
% toc=index             legt im Inhaltsverzeichnis einen Eintrag für das Stichwortverzeichnis an.
% toc=bibliography      legt im Inhaltsverzeichnis einen Eintrag für das Literaturverzeichnis an.
% english               englische Worte wie "Chapter" und "References".
% twoside               Beidseitiges Dokument, wie in einem Buch.


%If you don't want this fancyheaders comment out lines 17 to 24
% Fancyheader
\usepackage{fancyhdr}                   % Wie der Name schon sagt, um fancy Header zu generieren.
\pagestyle{fancy}                       % Fancy Header sollen angezeigt werden
\renewcommand{\sectionmark}[1]{\markright{\thesection.\ #1}{}}    % Verhindert dass rightmark ausschließlich Grußbuchstaben benutzt
\fancyhead[LE,RO]{\rightmark}           % Links bei geraden und rechts bei ungeraden Seitenzahlen soll der Name der Section stehen.
\fancyhead[LO,RE]{}                     % Links bei ungeraden und rechts bei geraden Seitenzahlen soll nichts stehen.
\fancyfoot[C]{}                         % Keine mittigen Seitenzahlen
\fancyfoot[LE,RO]{\thepage}             % Seitenzahlen unten in die jeweilige äußere Ecke





\setcounter{secnumdepth}{3}     % Nummerierungstiefe (chapter, section, subsection, ...).
\setcounter{tocdepth}{3}        % Nummerierungstiefe im Inhaltsverzeichn is.

\usepackage[linesnumbered,ruled,vlined]{algorithm2e}    % Algorithmen setzen.
\usepackage{amsmath,amssymb,amsthm,amsfonts,amsbsy,latexsym}    % "Notwendige" AMS-Math Pakete.
\usepackage{array}                      % Bessere Tabellen.
\renewcommand{\arraystretch}{1.15}      % Tabellen bekommen ein wenig mehr Platz.
\usepackage{bbm}                        % Dicke 1.
\usepackage[backend=biber, style=alphabetic]{biblatex}  % Gute Erweiterung zu bibtex, Wird für Referenzen benutzt.
\bibliography{thesis}   % Die verwendeten Referenzen (.bib-Datei)
\usepackage[hypcap]{caption}            % Damit Hyperrefs bei der figure-Umgebung auf die Figure zeigt statt auf die Caption.
\usepackage{datetime}                   % Um \today einzustellen.
\newdateformat{mydate}{\THEDAY{}th \monthname{} \THEYEAR{}}
\usepackage{diagbox}                    % Diagonale in Tabellen.
\usepackage{enumitem}                   % Zum Ändern der Nummerierungsumgenung 'enumerate'
\setlist[enumerate,1]{label=(\roman*)}  % Aufzählungen sind vom Typ 'Klammer auf; kleine römische Zahl; Klammer zu'
\usepackage[T1]{fontenc}                % Bessere Schrift
\usepackage{ifthen}                     % Zum checken ob Parameter leer sind.
\usepackage[utf8]{inputenc}             % utf8 als Eingabeformat.
\usepackage{lmodern}                    % Bessere Schrift
\usepackage{listings}                   % Code Listings.
\usepackage{mathtools}                  % Subscript unter Summen behandeln. Der Befehl lautet \mathclap.
\usepackage{makeidx}                    % Stichwortverzeichnis.
\makeindex                              % Stichwortverzeichnis erstellen.
\renewcommand{\indexname}{Index}        % Name des Index definieren.
\usepackage{multirow}                   % In Tabellen mehrere Zeilen zu einer machen.
\usepackage{rotating}                   % Um Figures zu drehen.
\usepackage{scrhack}                    % Verbessert die Zusammenarbeit von KOMA mit anderen Paketen (z.B, listing).
\usepackage{stackrel}                   % Symbole übereinander stapeln.
\usepackage[dvipsnames]{xcolor}         % Gefärbter Text und so.
\usepackage{tikz}                       % Graphen und kommutative Diagramme. Muss nach xcolor eingebunden werden.
\usepackage{tikz-cd}                    % Kommutative Diagramme.
\usepackage{transparent}                % Braucht mal manchmal für inkscape bilder.
\usetikzlibrary{patterns}               % Zu malen von schraffierten Flächen.

\graphicspath{{pictures/}}              % Pfad in dem die mit Inkscape erstellen Bilder liegen (relativ zum Hauptverzeichnis).

% Workaround, damit keine unnötigen Leerzeichen entstehen.
\let\oldindexdefn\index
\renewcommand*{\index}[1]{\oldindexdefn{#1}\ignorespaces}
\let\oldlabeldefn\label
\renewcommand*{\label}[1]{\oldlabeldefn{#1}\ignorespaces}

% Workaround, Linebreak nach ldots erlaubt.
\newcommand{\origldots}{}
\let\origldots\ldots
\renewcommand{\ldots}{\allowbreak\origldots}


% Symbolverzeichnis
\usepackage[intoc, english]{nomencl}     % Symbolverzeichnis.
% intoc                 die Symbolliste in das Inhaltsverzeichnis aufnehmen.
% english               englische Worte wie "Seite".
\renewcommand{\nomname}{Symbol Index}   % Definiert die Überschrift des Symbolverzeichnises.
\renewcommand{\nomlabelwidth}{80pt}     % Platz der einem Symbol gegönnt wird.
\newcommand{\symbolindex}[4][]{{\nomenclature[#1]{#2}{#3\ifthenelse{\equal{#4}{}}{}{ -- #4}\nomnorefpage}}\ignorespaces}    % Verbesserte Version von "\nomenclature". Erzeugt Symbol Beschreibung - Referenz.
\renewcommand*{\nompreamble}{\markright{\nomname}}    % Workaround: Fancyhdr schreibt im Symbolverzeichnis sonst den Namen des leztztes Kapitels.
\makenomenclature                       % Symbolverzeichnis erstellen.

% Anklickbare Referenzen (letztes eingebundenes Paket)
\usepackage{hyperref}                   % Referenzen innerhalb des Dokuments anklickbar machen. Achtung: Muss das letzte Paket im Präambel sein.
\hypersetup{                            % Optionen von hyperref Einstellen.
    colorlinks=true,                    % gefärbte Links an Stelle von Boxen.
    linkcolor=black,                     % Farbe interner Links.
    citecolor=black,                     % Farbe von Referenzen.
    urlcolor=black                       % Farbe von Internetlinks.
}

%       Makros und Definition der Titelseite
\makeatletter                           % Um Makros mit @ benutzen zu können. Zum Beispiel @author.
\newcommand*{\authora}[1]{\def\@authora{#1}}            % Autor
\newcommand*{\authorb}[1]{\def\@authorb{#1}}            % Autor.
\newcommand*{\geburtsdatuma}[1]{\def\@geburtsdatuma{#1}}        % Geburtsdatum.
\newcommand*{\geburtsdatumb}[1]{\def\@geburtsdatumb{#1}}        % Geburtsdatum.
\newcommand*{\geburtsorta}[1]{\def\@geburtsorta{#1}}    % Gebortsort.
\newcommand*{\geburtsortb}[1]{\def\@geburtsortb{#1}}    % Gebortsort.
\newcommand*{\thesentyp}[1]{\def\@thesentyp{#1}}        % Thesentyp z.B. Bachelorarbeit oder Master's Theses.
\newcommand*{\betreuera}[1]{\def\@betreuera{#1}}          % Betreuer z.B. Lieber Gott
\newcommand*{\betreuerb}[1]{\def\@betreuerb{#1}}          % zweiter Betreuer z.B. Jesus
\newcommand*{\institut}[1]{\def\@institut{#1}}          % Institut z.B. Mensa

% Definition der Titelseite.
\renewcommand\maketitle{
\KOMAoptions{twoside = false}   % Die Titelseite soll (bis auf Bindekorrektur) mittig auf dem Papier erscheinen.
\begin{titlepage}       % Definiere die Titelseite wie folgt.
    \vspace*{\fill}             % Um den Inhalt der Titelseite vertikal zu zentrieren, benutzen wir \vspace*{\fill} [Inhalt] \vspace*{\fill}.
    \begin{center}
        \parbox{14cm}{\centering\huge\bfseries \@title \par}\\\vspace{2em}      % Mache eine Paragraphenbox (Titel kann in mehreren Zeilen stehen) der Breite 15cm. Der Inhalt ist zentriert, sehr groß geschrieben, fett und mit Serifen. Anschließend 2 Einheiten Platz.
        {\Large\@authora}\\\vspace{.3em}                                        % Name des Authors in großer Schrift. Anschließend eine wenig Platz.
        Born   \@geburtsdatuma \ in \@geburtsorta\\\vspace{2em}                 % Geburtsdatum und Ort. Anschließend zwei Einheit Platz.
        
        {\large \@date}\\\vspace{15em}                                          % Datum in mittelgroßer Schrift. Anschließend 15 Einheiten Platz.
        {\large \@thesentyp}\\\vspace{1em}                                      % Der Text 'Master's Thesis  Mathematics' in großer Schrift. Anschließend eine Einheit Platz.
        {\large \@betreuera}\\\vspace{1em}  
         {\large \@betreuerb}\\\vspace{1em} 
        % Der Text 'Advisor: Prof. Dr. Carl-Friedrich Bödigheimer' in großer Schrift. Anschließend eine Einheit Platz.
        {\large\scshape \@institut}\\\vspace{9em}                                    % Der Text 'Mathematisches Institut' in großer 'small caps'-Schrift. Anschließend neun Einheit Platz.
        {\large\scshape Mathematisch-Naturwissenschaftliche Fakultät der}\\\vspace{1em}      % Der Text 'Mathematisch-Naturwissenschaftliche Fakultät der' in großer 'small caps'-Schrift. Anschließend eine Einheit Platz.
        {\large\scshape Rheinischen Friedrich-Wilhelms-Universität Bonn}             % Der Text 'Rheinischen Friedrich-Wilhelms-Universität Bonn' in großer 'small caps'-Schrift.
    \end{center}
    \vspace*{\fill}             % Hier endet der Inhalt des Titelblattes.
    \clearpage                  % Beende die Seite.
    
    \thispagestyle{empty}       % Die zweite Seite einfach nur leer sein. Dazu muss sie frei von Überschriften / Seitenzahlen / ... sein.
    \null\clearpage             % Beende die Seite. (Dazu ist mindestens ein Zeichen notwendig und dafür verwenden wir das Makro \null.)
\end{titlepage}         % Ende der Definition der Titelseite.
\KOMAoptions{twoside}   % Ab hier wieder twoside.
% Räume Makors auf damit ein weiteres \maketitel, \author, ... nichts mehr produziert.
\global\let\maketitle\relax
\global\let\@authora\@empty
\global\let\@authorb\@empty
\global\let\@geburtsdatuma\@empty
\global\let\@geburtsdatumb\@empty
\global\let\@geburtsorta\@empty
\global\let\@geburtsortb\@empty
\global\let\@thesentyp\@empty
\global\let\@betreuera\@empty
\global\let\@betreuerb\@empty
\global\let\@institut\@empty
\global\let\@date\@empty
\global\let\@title\@empty
\global\let\title\relax
\global\let\authora\relax
\global\let\authorb\relax
\global\let\geburtsdatuma\relax
\global\let\geburtsdatumb\relax
\global\let\geburtsorta\relax
\global\let\geburtsortb\relax
\global\let\thesentyp\relax
\global\let\betreuera\relax
\global\let\betreuerb\relax
\global\let\institut\relax
\global\let\date\relax
\global\let\and\relax
}
\makeatother
%       Ende Makros und Definition der Titelseite


\authora{your name}           % Verfasser 1
\geburtsdatuma{your birthday}      % Geburtsdatum.
\geburtsorta{your birthplace}                  % Gebortsort.
\thesentyp{Master's Thesis  Mathematics}        % Thesentyp.
\betreuera{Advisor: Prof.\ Dr.\ Your Advisor}        % Betreuer
\betreuerb{Second Advisor: Prof.\ Dr.\ Your Advisor} 
%Second Advisor
\institut{Institute for Applied Mathematics}      % Institut an dem die Arbeit angefertigt wurde
\date{12th September 2014\\Last update: \mydate\today}              % Datum der Abgabe der Arbeit.
\title{Title}        % Titel der Masterarbeit.


%       Theoreme
\theoremstyle{definition}               % Name: dick            Text: normal.
\newtheorem{defi}{Definition}[section]  % Der Zähler ist defi = Sectionzähler.1 . Sectionzähler soll bei Benutzung von defi nicht erhöht werden.
\newtheorem*{defi*}{Definition}
\newtheorem{example}[defi]{Example}
\newtheorem{notation}[defi]{Notation}
\newtheorem{rem}[defi]{Remark}
\newtheorem{defcor}[defi]{Definition/Corollary}
\newtheorem{defprop}[defi]{Definition/Proposition}
\newtheorem{defthm}[defi]{Definition/Theorem}

\theoremstyle{plain}
\newtheorem*{conj}{Conjecture}
\newtheorem{cor}[defi]{Corollary}
\newtheorem{lem}[defi]{Lemma}
\newtheorem{prop}[defi]{Proposition}
\newtheorem*{prop*}{Proposition}
\newtheorem{thm}[defi]{Theorem}
\newtheorem*{thm*}{Theorem}


%       Makros
\newcommand{\bA}{\mathbb{A}}
\newcommand{\bB}{\mathbb{B}}
\newcommand{\bC}{\mathbb{C}}
\newcommand{\bD}{\mathbb{D}}
\newcommand{\bE}{\mathbb{E}}
\newcommand{\bF}{\mathbb{F}}
\newcommand{\bG}{\mathbb{G}}
\newcommand{\bH}{\mathbb{H}}
\newcommand{\bI}{\mathbb{I}}
\newcommand{\bJ}{\mathbb{J}}
\newcommand{\bK}{\mathbb{K}}
\newcommand{\bL}{\mathbb{L}}
\newcommand{\bM}{\mathbb{M}}
\newcommand{\bN}{\mathbb{N}}
\newcommand{\bO}{\mathbb{O}}
\newcommand{\bP}{\mathbb{P}}
\newcommand{\bQ}{\mathbb{Q}}
\newcommand{\bR}{\mathbb{R}}
\newcommand{\bS}{\mathbb{S}}
\newcommand{\bT}{\mathbb{T}}
\newcommand{\bU}{\mathbb{U}}
\newcommand{\bV}{\mathbb{V}}
\newcommand{\bW}{\mathbb{W}}
\newcommand{\bX}{\mathbb{X}}
\newcommand{\bY}{\mathbb{Y}}
\newcommand{\bZ}{\mathbb{Z}}

\newcommand{\pr}[2]{\text{pr}_{#1}(#2)} %projection
\newcommand{\compl}[1]{#1^{c}} %complement
\newcommand{\id}{\text{id}} %identity


\begin{document}
% Titelseite
\maketitle              % Titelseite ausgeben
\setcounter{page}{3}    % Die Titelseite und die darauffolgende leere Seite sollen gefälligst Seite 1 und 2 sein.
\tableofcontents        % Inhaltsverzeichnis ausgeben

% Einleitung
\cleardoublepage        % Kapitel immer rechts beginnen
\chapter*{Introduction}
\addcontentsline{toc}{chapter}{Introduction}

Lean is a programming language that is frequently used as a theorem prover. 
It was primarily developed by Leonardo de Moura who co-funded the Lean focused research organisation that has taken on the development for five years in 2023 \cite{LeanFRO2024}. 
The latest version is called Lean 4.
More about the technical details of Lean 4 can be found in \cite{deMoura2021}.

Lean is known for its extensive mathematical library called \emph{mathlib} of which the development has been largely community driven. 
Its github repository has just over 300 different contributors and multiple new pull requests every day that get approved or rejected by the 28 maintainers. 

There have been multiple large formalizations based on mathlib. 
Here are two examples: 
In the \emph{Liquid Tensor Experiment}, given to the Lean community by Peter Scholze as a challenge, Johan Commelin, Adam Topaz and a number of other contributors formalised a theorem by Peter Scholze and Dustin Clausen from condensed mathematics \cite{Commelin2022}.
Floris van Doorn, Patrick Massot and Oliver Nash have formalised the existence of sphere eversions, a concept from differential topology, showing that geometric areas of mathematics can also be successfully formalised in Lean \cite{vanDoorn2023}. 

There are also several large scale ongoing projects of which we again present two examples: 
Floris van Doorn is currently leading a a formalisation of a generalisation of Carleson's theorem, a theorem from fourier analysis, by Christoph Thiele and his group \cite{Becker2024}.
Additionally there is a project led by Kevin Buzzard that aims to reduce the famous Fermat's Last Theorem to mathematical facts already known by mathematicians in the 1980s, a starting point similar to that of Andrew Wiles and Richard Taylor, who first proved this theorem in 1995 \cite{Buzzard2024}.

One important concept that is currently missing in mathlib is CW-complexes. 
They were first invented by \Citeauthor{Whitehead2018} in 1949 in \cite{Whitehead2018} to state and prove the famous Whitehead theorem which says that a continuous map between CW-complexes that induces isomorphisms on all homotopy groups is a homotopy equivalence.
CW-complexes are especially useful when doing calculations for example of singular homology and cohomology. 
One reason is that their skeletal structure allows you to use induction.
Since we are interested in providing a basic theory of CW-complexes we will not focus on applications but instead on basic properties. 
An introduction to CW-complexes and their applications can be found in \cite{Lundell1969}.
Our mathematical discussion will mostly be based on \cite{Hatcher2001}.
In chapter 1 we will discuss CW-complexes from a purely mathematical perspective. 
Chapter 2 gives a short introduction to some aspects of Lean that will be useful to understand the formalisation of most of the content of chapter 1 which we will cover in chapter 3. 
Note that the focus of this thesis is the formalisation of CW-complexes. 
The accompanying code can be found at \url{https://github.com/scholzhannah/CWComplexes}.

% Modelle für den Modulraum
\cleardoublepage        % Kapitel immer rechts beginnen
\chapter{One}
In this chapter we will learn some useful tools.\\
\section{How to cross reference}
In this section we will learn to cross reference.\\
For this just use command \ref{introduction:first_section}.  You previously need to create a label.\\
You can also reference equations in this way
\begin{equation}
\label{eq:1}
\langle v,\text{Re}A v\rangle =\langle v,U^*DUv\rangle=\langle Uv,DUv\rangle\geq \lambda_1\|Uv\|^2=\lambda_1\|v\|^2
\end{equation}

\begin{lem}
\label{lem:1}
Let $A\in \mathbb{C}^{M\times M}$ be diagonally dominant then $A$ is invertible.
\end{lem}

This equations can be accsessed by \ref{eq:1} and \ref{lem:1}
\section{How to create Index}

In this section we will learn to add elements to the  index.\\
Just use the command as in the example.
\begin{defi}
\index{vectorspace}
    A vectorspace is...
\end{defi}

\section{How to create symbol index}
In this section we will learn to add elements to the symbol index.\\
Just use the command as in the example.

\nomenclature{$\bR$}{The real numbers}


% product of CW-complexes
\cleardoublepage        % Kapitel immer rechts beginnen
\chapter{Product}
In this chapter we will talk about the product.
We assume all spaces to be Hausdorff.

\section{K-spaces and the k-ification}

\begin{lem}
    Let $X$ be a k-space.
    Then the topologies of $X$ and $X_c$ coincide.
\end{lem}

\begin{lem}
    Let $X$ be an anti-compact T$_1$ space.
    Then $X_c$ has discrete topology.
\end{lem}
\begin{proof}
    Let $A \subseteq X_c$ be any set. We need to show that it is open. 
    By the definition of the k-ification it is enough to show that $A \cap C$
    is open in $C$ for every compact set $C \subseteq X$. 
    Since $X$ is anti-compact $C$ is finite.
    And by T$_1$ every finite set has discrete topology. 
    Thus $A \cap C$ is open in $C$ and $X_c$ has discrete topology.
\end{proof}

\begin{cor}
    Let $X$ be a non-discrete anti-compact T$_1$ space. 
    Then $X$ is not a k-space.
\end{cor}
\section{The product of CW-complexes}

\begin{lem}
    $(X \times Y)_c$ has weak topology,
    i.e. $A \subseteq (X \times Y)_c$ is closed iff $(Q_i^n \times P_j^m)(D^{n + m}) \cap A$ is closed for all $n, m \in \bN$, $i \in I_n$ and $j \in J_m$.
\end{lem}
\begin{proof}~
    \begin{enumerate}
        \item["$\Rightarrow$"] Since $D^{n + m}$ is compact, its image is compact and therefore closed. As the intersection of two closed sets $(Q_i^n \times P_j^m)(D^{n + m}) \cap A$ is closed as well.
        \item["$\Leftarrow$"] We know by definition of the k-ification that $A$ is closed if for every compact set $C \subseteq X \times Y$ $A \cap C$ is closed in $C$.
        Take such a compact set $C$.
        The projections $\pr{1}{C}$ and $\pr{2}{C}$ are compact as images of a compact set. 
        By ? there are finite sets $E \subseteq \{e_i^n \mid n \in \bN, i \in I_n \}$ and $F \subseteq \{f_j^m \mid m \in \bN, j \in J_m \}$ s.t $\pr{1}{C} \subseteq \bigcup_{e \in E} e$ and $\pr{2}{C} \subseteq \bigcup_{f \in F} f$.
        Thus 
        \[C \subseteq \pr{1}{C} \times \pr{2}{C} \subseteq \bigcup_{e \in E} e \times \bigcup_{f \in F} f = \bigcup_{e \in E} \bigcup_{f \in F} e \times f.\] 
        So $C$ is included in a finite union of cells of $(X \times Y)_c$. 
        Therefore 
        \[A \cap C = A \cap \left (\bigcup_{e \in E} \bigcup_{f \in F} e \times f \right )\cap C = \left (\bigcup_{e \in E} \bigcup_{f \in F} A \cap (e \times f)\right ) \cap C\] 
        is closed since by assumption $A \cap (e \times f)$ is closed for every $e$ and $f$ and the union is finite. Thus $A \cap C$ is in particular closed in $C$.
    \qedhere
    \end{enumerate}
\end{proof}


\cleardoublepage
\appendix
\chapter*{Appendix}
\addcontentsline{toc}{chapter}{Appendix}
\renewcommand{\thesection}{\Alph{section}}
\input{chapter_appendix/section_appendix_a}
\input{chapter_appendix/section_appendix_b}

% Weitere Symbole für das Symbolverzeichnis
\symbolindex[f]{$F$}{A topological or Riemann surface.}{}

% Symbolverzeichnis
\cleardoublepage        % Auch diese sollen auf der rechten Seite beginnen
\printnomenclature      % Symbolverzeichnis ausgeben

% Stichwortverzeichnis
\cleardoublepage        % Auch diese sollen auf der rechten Seite beginnen
\printindex             % Stichwortverzeichnis ausgeben

% Referenzen
\nocite{*}              % Alle Einträge der Bib-Datei sollen in die Referenzen
\cleardoublepage        % Auch diese sollen auf der rechten Seite beginnen
\printbibliography      % Bibliographie ausgeben.

\end{document}