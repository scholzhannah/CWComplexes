\documentclass[paper=a4, fontsize=11pt, BCOR=13mm, DIV=13, headinclude, toc=index, toc=bibliography, english, twoside]{scrreprt}
% Die verwendete Dokumentenklasse ist scrreprt. Die verwendeten Optionen sind:
%
% paper=a4              Papier ist a4.
% fontsize=11pt         Schrifgröße ist 11.
% DIV=13                Das Papier wird in d viele Spalten und d' viele Zeilen eingeteilt. Die Werte werden aus DIV berechnet.
% BCOR=1cm              Definiert den Patz, der auf der Innenseite beim Binden verloren geht.
% headinclude           sorgt dafür, dass genug Platz für die Header vorhanden ist.
% toc=index             legt im Inhaltsverzeichnis einen Eintrag für das Stichwortverzeichnis an.
% toc=bibliography      legt im Inhaltsverzeichnis einen Eintrag für das Literaturverzeichnis an.
% english               englische Worte wie "Chapter" und "References".
% twoside               Beidseitiges Dokument, wie in einem Buch.


%If you don't want this fancyheaders comment out lines 17 to 24
% Fancyheader
\usepackage{fancyhdr}                   % Wie der Name schon sagt, um fancy Header zu generieren.
\pagestyle{fancy}                       % Fancy Header sollen angezeigt werden
\renewcommand{\sectionmark}[1]{\markright{\thesection.\ #1}{}}    % Verhindert dass rightmark ausschließlich Grußbuchstaben benutzt
\fancyhead[LE,RO]{\rightmark}           % Links bei geraden und rechts bei ungeraden Seitenzahlen soll der Name der Section stehen.
\fancyhead[LO,RE]{}                     % Links bei ungeraden und rechts bei geraden Seitenzahlen soll nichts stehen.
\fancyfoot[C]{}                         % Keine mittigen Seitenzahlen
\fancyfoot[LE,RO]{\thepage}             % Seitenzahlen unten in die jeweilige äußere Ecke





\setcounter{secnumdepth}{3}     % Nummerierungstiefe (chapter, section, subsection, ...).
\setcounter{tocdepth}{3}        % Nummerierungstiefe im Inhaltsverzeichn is.

\usepackage[linesnumbered,ruled,vlined]{algorithm2e}    % Algorithmen setzen.
\usepackage{amsmath,amssymb,amsthm,amsfonts,amsbsy,latexsym}    % "Notwendige" AMS-Math Pakete.
\usepackage{array}                      % Bessere Tabellen.
\renewcommand{\arraystretch}{1.15}      % Tabellen bekommen ein wenig mehr Platz.
\usepackage{bbm}                        % Dicke 1.
\usepackage[backend=biber, style=alphabetic]{biblatex}  % Gute Erweiterung zu bibtex, Wird für Referenzen benutzt.
\bibliography{thesis}   % Die verwendeten Referenzen (.bib-Datei)
\usepackage[hypcap]{caption}            % Damit Hyperrefs bei der figure-Umgebung auf die Figure zeigt statt auf die Caption.
\usepackage[nodayofweek]{datetime}                   % Um \today einzustellen.
\newdateformat{mydate}{\THEDAY{}th \monthname{} \THEYEAR{}}
\usepackage{diagbox}                    % Diagonale in Tabellen.
\usepackage{enumitem}                   % Zum Ändern der Nummerierungsumgenung 'enumerate'
\setlist[enumerate,1]{label=(\roman*)}  % Aufzählungen sind vom Typ 'Klammer auf; kleine römische Zahl; Klammer zu'
\usepackage[T1]{fontenc}                % Bessere Schrift
\usepackage{ifthen}                     % Zum checken ob Parameter leer sind.
\usepackage{lmodern}                    % Bessere Schrift
\usepackage{listings}                   % Code Listings.
\usepackage{mathtools}                  % Subscript unter Summen behandeln. Der Befehl lautet \mathclap.
\usepackage{makeidx}                    % Stichwortverzeichnis.
\makeindex                              % Stichwortverzeichnis erstellen.
\renewcommand{\indexname}{Index}        % Name des Index definieren.
\usepackage{multirow}                   % In Tabellen mehrere Zeilen zu einer machen.
\usepackage{rotating}                   % Um Figures zu drehen.
\usepackage{scrhack}                    % Verbessert die Zusammenarbeit von KOMA mit anderen Paketen (z.B, listing).
\usepackage{stackrel}                   % Symbole übereinander stapeln.
\usepackage[dvipsnames]{xcolor}         % Gefärbter Text und so.
\usepackage{tikz}                       % Graphen und kommutative Diagramme. Muss nach xcolor eingebunden werden.
\usepackage{tikz-cd}                    % Kommutative Diagramme.
\usepackage{transparent}                % Braucht mal manchmal für inkscape bilder.
\usetikzlibrary{patterns}               % Zu malen von schraffierten Flächen.

\graphicspath{{pictures/}}              % Pfad in dem die mit Inkscape erstellen Bilder liegen (relativ zum Hauptverzeichnis).

% Workaround, damit keine unnötigen Leerzeichen entstehen.
\let\oldindexdefn\index
\renewcommand*{\index}[1]{\oldindexdefn{#1}\ignorespaces}
\let\oldlabeldefn\label
\renewcommand*{\label}[1]{\oldlabeldefn{#1}\ignorespaces}

% Workaround, Linebreak nach ldots erlaubt.
\newcommand{\origldots}{}
\let\origldots\ldots
\renewcommand{\ldots}{\allowbreak\origldots}

% to get upright greek letters
\usepackage{upgreek}

% To break urls correctly in bibliography
\setcounter{biburllcpenalty}{7000}
\setcounter{biburlucpenalty}{8000}

% to get rid of an error
\usepackage{bookmark}

%colors for lean syntax
\definecolor{keywordcolor}{rgb}{0.7, 0.1, 0.1}   % red
\definecolor{tacticcolor}{rgb}{0.0, 0.1, 0.6}    % blue
\definecolor{commentcolor}{rgb}{0.4, 0.4, 0.4}   % grey
\definecolor{symbolcolor}{rgb}{0.0, 0.1, 0.6}    % blue
\definecolor{sortcolor}{rgb}{0.1, 0.5, 0.1}      % green
\definecolor{attributecolor}{rgb}{0.7, 0.1, 0.1} % red

% To display lean code
\def\lstlanguagefiles{lstlean.tex}
% set default language
\lstset{language=lean, extendedchars=true}

%to get code in footnotes write \cprotect\footnote 
\usepackage{cprotect}


% Symbolverzeichnis
\usepackage[intoc, english]{nomencl}     % Symbolverzeichnis.
% intoc                 die Symbolliste in das Inhaltsverzeichnis aufnehmen.
% english               englische Worte wie "Seite".
\renewcommand{\nomname}{Symbol Index}   % Definiert die Überschrift des Symbolverzeichnises.
\renewcommand{\nomlabelwidth}{80pt}     % Platz der einem Symbol gegönnt wird.
\newcommand{\symbolindex}[4][]{{\nomenclature[#1]{#2}{#3\ifthenelse{\equal{#4}{}}{}{ -- #4}\nomnorefpage}}\ignorespaces}    % Verbesserte Version von "\nomenclature". Erzeugt Symbol Beschreibung - Referenz.
\renewcommand*{\nompreamble}{\markright{\nomname}}    % Workaround: Fancyhdr schreibt im Symbolverzeichnis sonst den Namen des leztztes Kapitels.
\makenomenclature                       % Symbolverzeichnis erstellen.

\usepackage{MA_Titlepage}

% Anklickbare Referenzen (letztes eingebundenes Paket)
\usepackage{hyperref}                   % Referenzen innerhalb des Dokuments anklickbar machen. Achtung: Muss das letzte Paket im Präambel sein.
\hypersetup{                            % Optionen von hyperref Einstellen.
    colorlinks=true,                    % gefärbte Links an Stelle von Boxen.
    linkcolor=black,                     % Farbe interner Links.
    citecolor=black,                     % Farbe von Referenzen.
    urlcolor=black                       % Farbe von Internetlinks.
}



%Name of the author of the thesis 
\authornew{Hannah Scholz}
%Date of birth of the Author
\geburtsdatum{\formatdate{21}{7}{2003}}
%Place of Birth
\geburtsort{Bonn, Germany}
%Date of submission of the thesis
\date{\formatdate{22}{8}{2024}}

%Name of the Advisor
% z.B.: Prof. Dr. Peter Koepke
\betreuer{Advisor: Prof. Dr. Floris van Doorn}
%name of the second advisor of the thesis
\zweitgutachter{Second Advisor: Prof. Dr. Philipp Hieronymi}

%Name of the Insitute of the advisor
%z.B.: Mathematisches Institut
%\institut{Institut XYZ}
\institut{Mathematisches Institut}
%\institut{Institut f\"ur Angewandte Mathematik}
%\institut{Institut f\"ur Numerische Simulation}
%\institut{Forschungsinstitut f\"ur Diskrete Mathematik}
%Title of the thesis 
\title{Formalisation of CW-complexes}
%Do not change!
\ausarbeitungstyp{Bachelor's Thesis  Mathematics}

%       Theoreme
\theoremstyle{definition}               % Name: dick            Text: normal.
\newtheorem{defi}{Definition}[section]  % Der Zähler ist defi = Sectionzähler.1 . Sectionzähler soll bei Benutzung von defi nicht erhöht werden.
\newtheorem*{defi*}{Definition}
\newtheorem{example}[defi]{Example}
\newtheorem{notation}[defi]{Notation}
\newtheorem{rem}[defi]{Remark}
\newtheorem{defcor}[defi]{Definition/Corollary}
\newtheorem{defprop}[defi]{Definition/Proposition}
\newtheorem{defthm}[defi]{Definition/Theorem}

\theoremstyle{plain}
\newtheorem*{conj}{Conjecture}
\newtheorem{cor}[defi]{Corollary}
\newtheorem{lem}[defi]{Lemma}
\newtheorem{prop}[defi]{Proposition}
\newtheorem*{prop*}{Proposition}
\newtheorem{thm}[defi]{Theorem}
\newtheorem*{thm*}{Theorem}


%       Makros
\newcommand{\bA}{\mathbb{A}}
\newcommand{\bB}{\mathbb{B}}
\newcommand{\bC}{\mathbb{C}}
\newcommand{\bD}{\mathbb{D}}
\newcommand{\bE}{\mathbb{E}}
\newcommand{\bF}{\mathbb{F}}
\newcommand{\bG}{\mathbb{G}}
\newcommand{\bH}{\mathbb{H}}
\newcommand{\bI}{\mathbb{I}}
\newcommand{\bJ}{\mathbb{J}}
\newcommand{\bK}{\mathbb{K}}
\newcommand{\bL}{\mathbb{L}}
\newcommand{\bM}{\mathbb{M}}
\newcommand{\bN}{\mathbb{N}}
\newcommand{\bO}{\mathbb{O}}
\newcommand{\bP}{\mathbb{P}}
\newcommand{\bQ}{\mathbb{Q}}
\newcommand{\bR}{\mathbb{R}}
\newcommand{\bS}{\mathbb{S}}
\newcommand{\bT}{\mathbb{T}}
\newcommand{\bU}{\mathbb{U}}
\newcommand{\bV}{\mathbb{V}}
\newcommand{\bW}{\mathbb{W}}
\newcommand{\bX}{\mathbb{X}}
\newcommand{\bY}{\mathbb{Y}}
\newcommand{\bZ}{\mathbb{Z}}

\newcommand{\fC}{\mathcal{C}}
\newcommand{\fL}{\mathcal{L}}
\newcommand{\fF}{\mathcal{F}}
\newcommand{\fR}{\mathcal{R}}
\newcommand{\fV}{\mathcal{V}}
\newcommand{\fM}{\mathcal{M}}
\newcommand{\fP}{\mathcal{P}}

\newcommand{\mathlib}{\texttt{Mathlib}\xspace}


\DeclareMathOperator{\aeq}{\equiv_{\alpha}} %alpha-equivalence
\newcommand{\alert}[1]{{\color{blue}{\relax\ifmmode\mathbf{#1}\else\textbf{#1}\fi}}}
\DeclareMathOperator{\cupdot}{\dot{\cup}} % Disjoint union
\DeclareMathOperator{\smodels}{\models_{\sigma}} %models with variable assignment
\DeclareMathOperator{\becont}{\triangleright_{\beta}} %beta contraction
\DeclareMathOperator{\bered}{\to_{\beta}} %beta reduction
\DeclareMathOperator{\beored}{\to_{\beta, 1}} %beta-one-reduction
\DeclareMathOperator{\beq}{\equiv_{\beta}} %beta-equivalence
\DeclareMathOperator{\ored}{\to_1} %one-reduction
\DeclareMathOperator{\iocont}{\triangleright_{\iota}} %iota contraction
\DeclareMathOperator{\iored}{\to_{\beta}} %iota reduction
\DeclareMathOperator{\ioored}{\to_{\beta, 1}} %iota-one-reduction
\DeclareMathOperator{\beiored}{\to_{\beta \iota}} %beta-iota reduction
\DeclareMathOperator{\beioored}{\to_{\beta \iota, 1}} %beta-iota-one-reduction


\newcommand{\pow}[1]{\fP(#1)} % power set
\newcommand{\fv}[1]{\text{fv}(#1)} %free variables
\newcommand{\abs}[1]{\lvert #1 \rvert} % abs value (underlying set of a model)
\newcommand{\interpret}[1]{\llbracket #1 \rrbracket} % interpretation in a model
\newcommand{\vect}[1]{\bar{#1}} % alternative vector line, not in use currently
\newcommand{\menquote}[1]{\ensuremath{\text{``} #1 \text{''}}} % quotes in math mode
\newcommand{\domain}[1]{\text{dom}(#1)} %domain
\newcommand{\down}{{\downarrow}} %downarrow without space around it
\newcommand{\up}{{\uparrow}} %uparrow without space around it
\newcommand{\suc}[1]{\text{Succ}(#1)} %successor function
\newcommand{\lamcur}{\lambda_{\to}^{\text{Curry}}} % Curry-style typed lambda calculus
\newcommand{\lamchu}{\lambda_{\to}^{\text{Church}}} % Church-style typed lambda calculus

\newcommand{\proj}[2]{\text{P}_{#1}^{#2}} %projection
\newcommand{\compl}[1]{#1^{c}} %complement
\newcommand{\id}{\text{id}} %identity
\newcommand{\preim}[2]{#1^{-1}(#2)} %preimage
\newcommand{\interior}[1]{\text{int}(#1)} %interior
\newcommand{\closure}[1]{\overline{#1}} %closure
\newcommand{\boundary}{\partial} %boundary
\newcommand{\restrict}[2]{\ensuremath{\left.#1\right|_{#2}}} %restriction
\newcommand{\norm}[1]{\left\lVert#1\right\rVert} %norm
\newcommand{\maximum}[2]{\text{max}(#1, #2)} %maximum





\begin{document}
% Titelseite
\maketitle              % Titelseite ausgeben
\setcounter{page}{3}    % Die Titelseite und die darauffolgende leere Seite sollen gefälligst Seite 1 und 2 sein.
\tableofcontents        % Inhaltsverzeichnis ausgeben

%introduction
\cleardoublepage
\chapter*{Introduction}
\addcontentsline{toc}{chapter}{Introduction}

Lean is a programming language that is frequently used as a theorem prover. 
It was primarily developed by Leonardo de Moura, who co-funded the Lean focused research organisation that has taken on the development for five years in 2023 \cite{LeanFRO2024}. 
The latest version is called Lean 4.
More about the technical details of Lean 4 can be found in \cite{deMoura2021}.

Lean is known for its extensive mathematical library called \emph{mathlib}, of which the development has been largely community driven. 
Its github repository has just over 300 different contributors and multiple new pull requests every day that get approved or rejected by the 28 maintainers. 

There have been multiple large formalisations based on mathlib. 
Here are two examples: 
In the \emph{Liquid Tensor Experiment}, given to the Lean community by Peter Scholze as a challenge, Johan Commelin, Adam Topaz and other contributors formalised a theorem by Peter Scholze and Dustin Clausen from condensed mathematics \cite{Commelin2022}.
In addition, Floris van Doorn, Patrick Massot and Oliver Nash have formalised the existence of sphere eversions, a concept from differential topology, showing that geometric areas of mathematics can also be successfully formalised in Lean \cite{vanDoorn2023}. 

There are also several large-scale ongoing projects of which we again present two examples: 
Floris van Doorn is currently leading a formalisation of a generalisation of Carleson's theorem, a theorem from fourier analysis, by Christoph Thiele and his collaborators \cite{Becker2024}.
Additionally, there is a project led by Kevin Buzzard that aims to reduce the renowned Fermat's Last Theorem to mathematical facts already known by mathematicians in the 1980s, a starting point similar to that of Andrew Wiles and Richard Taylor, who first proved this theorem in 1995 \cite{Buzzard2024}.

One important concept that is currently missing in mathlib is CW-complexes. 
They were first invented by \Citeauthor{Whitehead2018} in 1949 in \cite{Whitehead2018} to state and prove the famous Whitehead theorem, which says that a continuous map between CW-complexes that induces isomorphisms on all homotopy groups is a homotopy equivalence.
CW-complexes are especially useful when doing calculations, for example, of singular homology and cohomology. 
One reason is that their skeletal structure allows one to use induction.
Since we are interested in providing a basic theory of CW-complexes, we will not focus on applications but instead on basic properties. 
An introduction to CW-complexes and their applications can be found in \cite{Lundell1969}.
Our mathematical discussion will mostly be based on \cite{Hatcher2001}.
In chapter 1 we will discuss CW-complexes from a purely mathematical perspective. 
Chapter 2 gives a short introduction to some aspects of Lean that will be useful to understand the formalisation of most of the content of chapter 1 which we will cover in chapter 3. 
Note that the focus of this thesis is the formalisation of CW-complexes. 
The accompanying code can be found at \url{https://github.com/scholzhannah/CWComplexes}.

%mathematics of CW-complexes
\cleardoublepage 
\input{chapter_mathematics/chapter_mathematics.tex}

%lean and mathlib
\cleardoublepage
\chapter{Lean and mathlib}

In this chapter we will discuss some general concepts about Lean and its mathematical library mathlib. 
We will first explain the logic, i.e. the type theory, that is used in Lean. 
While it is helpful to know some theory about it, it is not necessary to understand the type theory in depth to formalise mathematics in Lean or read and understand the rest of this thesis. 
Most important are some constructions, that are explained at the end of the following section, and how you can use them in practice.

\section{The type theory of Lean}
\label{sec:typetheory}

\emph{Type theory} was first proposed by \Citeauthor{Russell1908} in \citeyear{Russell1908} \cite{Russell1908} as a way to axiomatise mathematics and resolve the paradoxes (most famously Russell's paradox) that were discussed at the time. 
While type theory has lost its relevance as a foundation of mathematics to set theory it has since been studied in both mathematics and computer science. 
It was first used in formal mathematics in 1967 in the formal language \emph{AUTOMATH}. 
More about the history of type theory can be found in \cite{Kamareddine2004}. 
Discussions of type theory in mathematics and especially its connections to homotopy theory forming the new area of \emph{homotopy type theory} can be found in \cite{hottbook}.
We will now focus on the type theory as used in Lean.
Its type theory along with the type theory of other proof assistants such as \emph{Coq} are based on \emph{constructive type theory} developed by Per Martin-Löf which makes use of dependant types \cite{Martin-Löf1984}.
A detailed account of Lean's type theory can be found in \cite{Carneiro2019}. 
The following short discussion is based on \cite{Avigad2024}. 

Lean uses what is called a \emph{dependent type theory}.
In type theory every object has a type. 
A type can for example be the natural numbers or propositions which we write in Lean as \lstinline{Nat} or \lstinline{ℕ} and \lstinline{Prop} respectively.
To assert that \lstinline{n} is a natural number or that \lstinline{p} is a proposition we write \lstinline{n : ℕ} and \lstinline{p : Prop}. 
Proofs of a proposition \lstinline{p} also form a type written as \lstinline{p}.
If you want to say that \lstinline{hp} is a proof of \lstinline{p} then you can simply write \lstinline{hp : p}.
Something to note about proofs in Lean is that contrary to other type theories the type theory of Lean has \emph{proof irrelevance} which means that two proofs of a proposition \lstinline{p} are by definition assumed to be the same.

Even types themselves have types. 
In Lean the type of natural numbers \lstinline{ℕ} has the type \lstinline{Type}. 
The type of propositions \lstinline{Prop} is also of type \lstinline{Type}.
As to not run into a paradox called Girard's paradox there is a hierarchy of types \cite{Coquand1986}. 
The type of \lstinline{Type} is \lstinline{Type 1}, the type of \lstinline{Type 1} is \lstinline{Type 2} and so on.
These are called type-universes. 
The notation \lstinline{α : Type*} is a way of stating that \lstinline{α} is a type in an arbitrary universe.

There are a few ways to construct new types from existing ones. 
Some of them are very similar to constructions on sets such as the cartesian product of types written as \lstinline{α × β} or the type of functions from \lstinline{α} to \lstinline{β} written as \lstinline{α → β} where \lstinline{α} and \lstinline{β} are types.
Elements of \lstinline{α × β} can be written as \lstinline{(a, b)} for every \lstinline{a : α} and \lstinline{b : β}.
Elements of \lstinline{α → β} can be written as \lstinline{fun a ↦ s a} for some \lstinline{s : α → β}. 
Since these are quite self-explanatory we will not go into more detail.
We will now mainly discuss constructions that do not fulfil this criterium. 
A first example is the sum type of two types \lstinline{α} and \lstinline{β} written as \lstinline{α ⊕ β} which is the equivalent to a disjoint union of sets. 
Elements of this type are of the form \lstinline{Sum.inl a} for \lstinline{a : α} or \lstinline{Sum.inr b} for \lstinline{b : β}.
When given an \lstinline{x : α ⊕ β} we can use the construction 

\begin{lstlisting}
match x with 
| Sum.inl a => ⋯ 
| Sum.inr b => ⋯
\end{lstlisting}

to write two different definitions or proofs depending on whether \lstinline{i} originates from an element of \lstinline{α} or \lstinline{β}. 
In this code snippet the names \lstinline{a} and \lstinline{b} are arbitrary.

The next two examples explain why this type theory is a dependent type theory:
If we have a type \lstinline{α} and for every \lstinline{(a : α)} a type \lstinline{β a} (i.e. \lstinline{β} is a function assigning a type to every \lstinline{a : α}) then we can construct the \emph{pi type} or dependent function type written as \lstinline{(a : α) → β a} or \lstinline{Π (a : α), β a}. 
We can construct an element of this type by writing \lstinline{fun a ↦ s a} for some \lstinline{s : (a : α) → β a}. 
Here is an example: 
Assume that you want a function that for every pair in a cartesian product \lstinline{α × α} for any type \lstinline{α} returns the first element. 
Then this would be a function that depends on \lstinline{α} and whose type is therefore the dependant function type \lstinline{(α : Type*) → α × α → α}.

The dependent version of the cartesian product is called a \emph{sigma type} and can be written as \lstinline{(a : α) × β a} or \lstinline{Σ a : α, β a} for \lstinline{α} and \lstinline{β} the same way as above. 
An element of the sigma type can be written as \lstinline{⟨a, b⟩} for \lstinline{a : α} and \lstinline{b : β a}. 
When given an  element of the sigma type \lstinline{x : Σ a : α, β a} one can write \lstinline{obtain ⟨a, b⟩ := x} to deconstruct \lstinline{x}.

\section{Implicit arguments and typeclass inference}
\label{sec:implicitandtypeclass}

A crucial factor that makes Lean more comfortable to use and makes the formalisation process feel closer to doing mathematics on paper is its use of \emph{implicit arguments} and \emph{typeclass inference}. 
We will explain both of these concepts in this section.

First let us discuss implicit arguments based on \cite{Avigad2024}. 
One way that we could define continuity in Lean is the following\cprotect\footnote{The code in this section will run if \lstinline{import Mathlib.Topology.MetricSpace.Basic} is written at the top of the file.}:

\begin{lstlisting}
structure Continuous' (X Y : Type*) (t : TopologicalSpace X) 
    (s : TopologicalSpace Y) (f : X → Y) : Prop where
  isOpen_preimage : ∀ s, IsOpen s → IsOpen (f ⁻¹' s)
\end{lstlisting}

Where a structure is a construct that can bundle both data and properties after the keyword \lstinline{where}. 
This structure has no data and one property which is named \lstinline{isOpen_preimage}. 
\lstinline{f ⁻¹' s} denotes the preimage of \lstinline{s} under \lstinline{f}.
But now if we are given two types \lstinline{X} and \lstinline{Y} with topologies \lstinline{s} and \lstinline{t} respectively and a map \lstinline{f : X → Y}, the statement that the map \lstinline{f} is continuous would be expressed in the following way:

\begin{lstlisting}
example (X Y : Type*) (t : TopologicalSpace X) (s : TopologicalSpace Y) 
  (f : X → Y) : Continuous' X Y t s f := ⋯
\end{lstlisting}

where everything before the colon is the context we described above and after the colon equal you could write a proof.

One thing that we can notice is that the types \lstinline{X} and \lstinline{Y} are contained in the definition of \lstinline{f} which means that Lean should be able to find that information itself. 
To tell Lean to do that you can replace the variables by underscores: 

\begin{lstlisting}
  example (X Y : Type*) (t : TopologicalSpace X) (s : TopologicalSpace Y) 
  (f : X → Y) : Continuous' _ _ t s f := ⋯
\end{lstlisting}

These two arguments are always clear from the context in this way. 
We therefore want to specify in the definition that they should not be given explicitly but instead inferred by the system.
We use curly brackets to do this: 

\begin{lstlisting}
structure Continuous'' {X Y : Type*} (t : TopologicalSpace X) 
    (s : TopologicalSpace Y) (f : X → Y) : Prop where
  isOpen_preimage : ∀ s, IsOpen s → IsOpen (f ⁻¹' s)
\end{lstlisting}

which enables us to write continuity like this: 

\begin{lstlisting}
example (X Y : Type*) (t : TopologicalSpace X) (s : TopologicalSpace Y) 
  (f : X → Y) : Continuous'' t s f := ⋯
\end{lstlisting}

This is already a lot shorter than what we had above but there is still room for improvement, as on paper we would probably just write "$f$ is continuous" since in most contexts $X$ and $Y$ will only have one specified topology each, that can be inferred by the reader.
The same thing is also true in Lean and we can achieve this by typeclass inference.
Typeclasses were first invented by \Citeauthor{Wadler1989} in \cite{Wadler1989} to be used in the programming language Haskell. 
They are a way to overload operations for various different types. 
For example, you might want to write code that works for all types that have a topology. 
In Lean this is possible by just stating that your input type \lstinline{X} is part of the typeclass \lstinline{TopologicalSpace}. 
You can specify that something ia a typeclass with the keyword \lstinline{class}. 
The definition of the typeclass of topological spaces in mathlib looks like this: 
\href{https://github.com/leanprover-community/mathlib4/blob/93828f4cd10fb8cab31700b110fd2751d36bf1b8/Mathlib/Topology/Defs/Basic.lean#L59-L71}{\faExternalLink}

\begin{lstlisting}
class TopologicalSpace (X : Type*) where
  protected IsOpen : Set X → Prop
  protected isOpen_univ : IsOpen univ
  protected isOpen_inter : ∀ s t, IsOpen s → IsOpen t → IsOpen (s ∩ t)
  protected isOpen_sUnion : ∀ s, (∀ t ∈ s, IsOpen t) → IsOpen (⋃₀ s)
\end{lstlisting}

Let us first explain what this code means: 
The keyword \lstinline{protected} means that these properties should not be accessed directly because there are lemmas that should be used instead. 
\lstinline{Set X} is the type that consists of all sets of elements of \lstinline{X}.
Thus the line \lstinline{protected IsOpen : Set X → Prop} expresses that \lstinline{IsOpen} is a property that can be assigned to a set in \lstinline{X}.
The rest of the lines discuss the properties of a topology.
\lstinline{univ} is the set that is composed of all elements of \lstinline{X} and \lstinline{⋃₀ s} is the union over the set \lstinline{s}. 
All of these explanations are not actually relevant to typeclasses, they are just for our understanding of the above code. 

Typeclasses are also expected to be inferred automatically. 
Local instances of these typeclasses can be written with square brackets, which tells Lean to infer these automatically.

We can now look at the version of continuity that is almost identical to that of mathlib: 
\href{https://github.com/leanprover-community/mathlib4/blob/93828f4cd10fb8cab31700b110fd2751d36bf1b8/Mathlib/Topology/Defs/Basic.lean#L138-L144}{\faExternalLink}

\begin{lstlisting}
structure Continuous {X Y : Type*} [t : TopologicalSpace X]
    [s : TopologicalSpace Y] (f : X → Y) : Prop where
  isOpen_preimage : ∀ s, IsOpen s → IsOpen (f ⁻¹' s)
\end{lstlisting}

which enables us to write that \lstinline{f} is continuous in the context explained above as follows:

\begin{lstlisting}
example (X Y : Type*) (t : TopologicalSpace X) (s : TopologicalSpace Y) 
    (f : X → Y) : Continuous f := ⋯
\end{lstlisting}

When you define a class you can then define instances of that class to be inferred whenever you talk you talk about a type with that instance. 
Mathlib defines the discrete topology on $\bZ$ as an instance: 
\href{https://github.com/leanprover-community/mathlib4/blob/93828f4cd10fb8cab31700b110fd2751d36bf1b8/Mathlib/Topology/Order.lean#L481-L481}{\faExternalLink}

\begin{lstlisting}
instance : TopologicalSpace ℤ := ⊥
\end{lstlisting}

where \lstinline{⊥} is the smallest element in the order that can be defined on the topologies of a space, i.e. the finest topology which is the discrete topology. 
That makes it so that for any map \lstinline{f : ℤ → ℤ} we can just write the following

\begin{lstlisting}
example (f : ℤ → ℤ) : Continuous f := ⋯
\end{lstlisting} 

and Lean automatically knows which topology we are talking about. 
We can additionally say that an instance implies another instance. 
If you have types \lstinline{X} and \lstinline{Y} which both have topologies defined on them this instance in mathlib gives you a topology on the product: 
\href{https://github.com/leanprover-community/mathlib4/blob/93828f4cd10fb8cab31700b110fd2751d36bf1b8/Mathlib/Topology/Constructions.lean#L50-L52}{\faExternalLink}

\begin{lstlisting}
instance instTopologicalSpaceProd {X Y : Type*} [t₁ : TopologicalSpace X] 
  [t₂ : TopologicalSpace Y] : TopologicalSpace (X × Y) := ⋯
\end{lstlisting}

which enables you to write the following

\begin{lstlisting}
example {X Y Z : Type*} [TopologicalSpace X] [TopologicalSpace Y]
  [TopologicalSpace Z] (f : X × Y → Z) : Continuous f := ⋯
\end{lstlisting}

This also works across different typeclasses. 
We can write 

\begin{lstlisting}
example {X : Type*} [MetricSpace X] (f : X → X) : Continuous f := ⋯
\end{lstlisting}

which works because Lean knows that a metric space is by definition a pseudometric space 
\href{https://github.com/leanprover-community/mathlib4/blob/93828f4cd10fb8cab31700b110fd2751d36bf1b8/Mathlib/Topology/MetricSpace/Defs.lean#L36-L38}{\faExternalLink}
which is a uniform space
\href{https://github.com/leanprover-community/mathlib4/blob/93828f4cd10fb8cab31700b110fd2751d36bf1b8/Mathlib/Topology/MetricSpace/Pseudo/Defs.lean#L100-L119}{\faExternalLink}
which is by definition a topological space
\href{https://github.com/leanprover-community/mathlib4/blob/93828f4cd10fb8cab31700b110fd2751d36bf1b8/Mathlib/Topology/UniformSpace/Basic.lean#L278-L289}{\faExternalLink}.

%formalisation 
\cleardoublepage
\input{chapter_formalisation/chapter_formalisation.tex}

%conclusion
\cleardoublepage 
\chapter*{Conclusion}
\addcontentsline{toc}{chapter}{Conclusion}

The aim of this thesis was to formalise the basic properties and some constructions of CW-complexes - a concept that is not yet in Lean's the mathematical library mathlib. 

We chose the historical definition to formalise the concept of CW-complexes in Lean: 

\begin{lstlisting}
class CWComplex.{u} {X : Type u} [TopologicalSpace X] (C : Set X) where
  cell (n : ℕ) : Type u
  map (n : ℕ) (i : cell n) : PartialEquiv (Fin n → ℝ) X
  source_eq (n : ℕ) (i : cell n) : (map n i).source = closedBall 0 1
  cont (n : ℕ) (i : cell n) : ContinuousOn (map n i) (closedBall 0 1)
  cont_symm (n : ℕ) (i : cell n) : ContinuousOn (map n i).symm (map n i).target
  pairwiseDisjoint' :
  (univ : Set (Σ n, cell n)).PairwiseDisjoint (fun ni ↦ map ni.1 ni.2 '' ball 0 1)
  mapsto (n : ℕ) (i : cell n) : ∃ I : Π m, Finset (cell m),
    MapsTo (map n i) (sphere 0 1) (⋃ (m < n) (j ∈ I m), map m j '' closedBall 0 1)
  closed' (A : Set X) (asubc : A ⊆ C) : IsClosed A ↔ ∀ n j, IsClosed (A ∩ map n j '' closedBall 0 1)
  union' : ⋃ (n : ℕ) (j : cell n), map n j '' closedBall 0 1 = C
\end{lstlisting}

One of the important properties that we were able to formalise is the relationship between compact sets and finite CW-complexes: 

\begin{lstlisting}
lemma compact_iff_finite : IsCompact C ↔ Finite C := ⋯

lemma compact_subset_finite_subcomplex {B : Set X} (compact : IsCompact B) :
    ∃ (E : Set X) (sub : Subcomplex C E), CWComplex.Finite (E ⇂ C) ∧ B ∩ C ⊆ E := 
  ⋯
\end{lstlisting}

Additionally we formalised the CW-complex structure on the product of two CW-complexes of which the more readable mathematical statement is the following: 

\begin{thm*}
    Let $X$, $Y$ be CW-complexes with families of characteristic maps $(Q_i^n \colon D_i^n \to X)_{n \in \bN, i \in I_n}$ and $(P_j^m \colon D_j^m \to Y)_{m \in \bN, j \in J_n}$. 
    Let $\openCell{n}{i}$ be the cells of $X$ and $\openCellf{m}{j}$ be the cells of $Y$.
    Then there is a CW-structure on $(X \times Y)_c$ with characteristic maps $(Q_i^n \times P_j^m \colon D_i^n \times D_j^m \to (X \times Y)_c)_{n,m \in \bN,i \in I_n,j \in J_m}$.
    The indexing sets $(K_l)_{l \in \bN}$ are given by $K_l = \bigcup_{n + m = l}I_n \times J_m$ for every $l \in \bN$ and the cells are therefore of the form $\openCell{n}{i} \times \openCellf{m}{j}$ for $n, m \in \bN$, $i \in I_n$ and $j \in J_m$.
\end{thm*}

Ultimately the goal is to add this work into mathlib so that others can build upon it. 
I have already started to contribute some of the auxiliary lemmas unrelated to CW-complexes that were needed along the way. 
There is still much that can be done: 
First of all the definition could be generalized to relative CW-complexes and one could implement the modern definition as well.
There are still some constructions that could be useful such as the quotient of a CW-complex by a subcomplex. 
More high level goals could be the Whitehead Theorem or cellular homology and cohomology. 

%summary 
\cleardoublepage
\chapter*{German summary}
\addcontentsline{toc}{chapter}{German summary}

Diese Arbeit befasst sich mit der Formalisierung von CW-Komplexen im Beweisassistenten Lean. 
Beweisassistenten können dazu genutzt werden, formal die Richtigkeit von Beweisen in einem logischen digitalen System zu überprüfen. 
Lean ist unter anderem wegen seiner umfangreichen mathematischen Bibliothek \emph{mathlib} ein sehr beliebter Beweisassistent. 
Ein Konzept, das in dieser Bibliothek jedoch noch fehlt, sind die CW-Komplexe. 
In der Topologie sind sie häufig hilfreich, um Berechnungen, zum Beispiel von Homologie und Kohomologie, zu vereinfachen. 

Im ersten Kapitel beschäftigen wir uns mit der mathematischen Theorie hinter den CW-Komplexen. 
Wir konzentrieren uns hierbei auf die historische und nicht die moderne Definition, da uns diese die Formalisierung erleichtert. 
Wir beweisen einige grundlegende Eigenschaften von CW-Komplexen und beschäftigen uns dann im Detail mit verschiedenen Konstruktionen. 
Besonders dem Produkt zweier CW-Komplexes widmen wir sehr viel Zeit: Wir zeigen an einem Gegenbeispiel, dass das Produkt nicht notwendigerweise wieder ein CW-Komplex sein muss, führen dann k-Räume ein und beweisen, dass die k-ifizierung eines Produktes von zwei CW-Komplexen immer ein CW-Komplex ist.

Im zweiten Kapitel behandeln wir kurz drei technische Details von Lean: Die Typentheorie, d.h. die zugrundeliegende Logik, von Lean, implizite Argumente und Typklasseninferenz.
Diese Inhalte sind interessante Zusatzinformation, aber nicht unbedingt notwendig für das Verständnis der Arbeit. 

Im dritten Kapitel beschäftigen wir uns dann mit der Formalisierung von CW-Komplexen in Lean. 
Wir besprechen, welche Designentscheidungen getroffen wurden und warum, und machen auf Unterschiede in der Formalisierung aufmerksam. 
Dabei zeigen und erklären wir Ausschnitte des Codes. 
Wir haben einen Großteil des Inhalts des ersten Kapitels in Lean formalisiert, unter anderem den Zusammenhang zwischen endlichen Unterkomplexen und kompakten Mengen und die CW-Komplex-Struktur auf der k-ifizierung des Produktes von CW-Komplexen.
Den kompletten Code findet man unter \url{https://github.com/scholzhannah/CWComplexes}.


% Symbolverzeichnis
\cleardoublepage        % Auch diese sollen auf der rechten Seite beginnen
\printnomenclature      % Symbolverzeichnis ausgeben

% Referenzen
\nocite{*}              % Alle Einträge der Bib-Datei sollen in die Referenzen
\cleardoublepage        % Auch diese sollen auf der rechten Seite beginnen
\printbibliography      % Bibliographie ausgeben.

\end{document}