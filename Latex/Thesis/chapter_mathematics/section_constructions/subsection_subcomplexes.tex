\subsection{Subcomplexes}

One important way to get a new CW-complex from an existing one is to consider subcomplexes which we will discuss in this section. 

Let $X$ be a CW-complex. A subcomplex of $X$ is defined as follows:

\begin{defi} \label{defi:subcomplex}
    A subcomplex of $X$ is a set $E \subseteq X$ together with a set $J_n \subseteq I_n$ for every $n \in \bN$ such that:
    \begin{enumerate}
        \item $E$ is closed.
        \item $\bigcup_{n \in \bN} \bigcup_{i \in J_n} \openCell{n}{i} = E$.
    \end{enumerate}
\end{defi}

Note that here we want $E$ to be the union of the open cells instead of the union of the closed cells as in definition \ref{defi:CWComplex2}. 
This is just to make some of the proofs easier and we will get the other version once we prove that every subcomplex is again a CW-complex. 

Here are some alternative ways to define subcomplexes. 
These are taken from chapter 7.4 in \cite{Jänich2001}.
The proof that these three notions are equivalent can be found in there. 
We will just show the direction that is useful to us. 

\begin{lem} \label{lem:subcomplex2}
    Let $E \subseteq X$ and $J_n \subseteq I_n$ for $n \in \bN$ be such that 
    \begin{enumerate}
        \item For every $n \in \bN$ and $i \in I_n$ we have $\closedCell{n}{i} \subseteq E$. 
        \item $\bigcup_{n \in \bN} \bigcup_{i \in J_n} \openCell{n}{i} = E$.
    \end{enumerate}
    Then $E$ is a subcomplex of $X$.
\end{lem}
\begin{proof}
    Property (ii) in definition \ref{defi:subcomplex} is clear immediately. 
    So we only need to show that $E$ is closed. 
    We apply lemma \ref{lem:closediffinteropenorclosed} which means we only need to show that for every $n \in \bN$ and $i \in I_n$ either $E \cap \closedCell{n}{i}$ or $E \cap \openCell{n}{i}$ is closed. 
    So let $n \in \bN$ and $i \in I_n$. 
    We differentiate the cases $i \in J_n$ and $i \notin J_n$.
    For the first one notice that by property (i) $E$ can be expressed as a union of closed cells: $E = \bigcup_{m \in \bN} \bigcup_{j \in J_n} \openCell{m}{j} \subseteq \bigcup_{m \in \bN} \bigcup_{j \in J_n} \closedCell{m}{j} \subseteq E$. 
    This gives us $E \cap \closedCell{n}{i} = \closedCell{n}{i}$ which is closed by lemma \ref{lem:closedCellclosed}. 
    Now for the case $i \notin J_n$ the disjointness of the open cells of $X$ gives us that $E \cap \openCell{n}{i} = (\bigcup_{m \in \bN} \bigcup_{j \in J_n} \openCell{m}{j}) \cap \openCell{n}{i} = \varnothing$ which is obviously closed. 
\end{proof}

And here is a third way to express the property of being a subcomplex:

\begin{lem}
    Let $E \subseteq X$ and $J_n \subseteq I_n$ for $n \in \bN$ be such that 
    \begin{enumerate}
        \item $E$ is a CW-complex with respect to the cells determined by $X$ and $J_n$.
        \item $\bigcup_{n \in \bN} \bigcup_{i \in J_n} \openCell{n}{i} = E$.
    \end{enumerate}
    Then $E$ is a subcomplex of $X$.
\end{lem}
\begin{proof}
    We will show that this satisfies the properties of the construction above in lemma \ref{lem:subcomplex2}.
    Property (ii) is again immediate. 
    Property (i) combined with the definition \ref{defi:CWComplex2} of a CW-complex immediately gives us property (i) of lemma \ref{lem:subcomplex2}.
\end{proof}

Now we can show that a subcomplex is indeed again a CW-complex: 

\begin{lem}
    Let $E \subseteq X$ together with $J_n \subseteq I_n$ for every $n \in \bN$ be a subcomplex of the CW-complex $X$. 
    Then $E$ is again a CW-complex with respect to the cells determined by $J_n$ and $X$.
\end{lem}
\begin{proof}
    We show this by verifying the properties in the definition \ref{defi:CWComplex2} of a CW-complex. 
    
\end{proof}