\chapter*{Introduction}
\addcontentsline{toc}{chapter}{Introduction}


Theorem provers are used to formally verify proofs using strict logical frameworks in digital systems. 
They can help ensure that every detail of a proof is indeed correct and their libraries provide correct and connected accounts of mathematical theories with complete proofs.

The programming language and proof assistant Lean is among these theorem provers.
Its extensive mathematical library \emph{mathlib}, of which the development is largely community driven, has made it, among other reasons, a popular theorem prover both for students contributing small amounts of work as mathematical side projects and also for scientists who specialise in formalisation to manage ambitious projects with many contributors. 
Mathlib itself could be considered one such project. 
This thesis aims to contribute to and build upon this enormous amount of previous work by formalising CW-complexes in Lean, a concept that is not yet part of mathlib.

Lean itself was primarily developed by Leonardo de Moura, who co-founded the Lean focused research organisation that has taken on the development for five years in 2023 \cite{LeanFRO2024}. 
The latest version is called Lean 4.
More about the technical details of Lean 4 can be found in \cite{deMoura2021}.

The accompanying library mathlib is available on GitHub at \url{https://github.com/leanprover-community/mathlib4}. 
This repository has just over 300 different contributors and multiple new pull requests every day that get approved or rejected by the 28 maintainers.
While mathlib is largely focused on providing a cohesive system of foundational mathematical theories, there have been a multiple large formalisations of advanced mathematical content based on mathlib, which also contributed to the library along the way. 

Here are two examples: 
In the \emph{Liquid Tensor Experiment}, given to the Lean community by Peter Scholze as a challenge, Johan Commelin, Adam Topaz and other contributors formalised a theorem by Peter Scholze and Dustin Clausen from condensed mathematics \cite{Commelin2022}.
In addition, Floris van Doorn, Patrick Massot and Oliver Nash have formalised the existence of sphere eversions, a concept from differential topology, showing that geometric areas of mathematics can also be successfully formalised in Lean \cite{vanDoorn2023}. 

There are also several large-scale ongoing projects of which we again present two examples: 
Floris van Doorn is currently leading a formalisation of a generalisation of Carleson's theorem, a theorem from fourier analysis, by Christoph Thiele and his collaborators \cite{Becker2024}.
Additionally, there is a project led by Kevin Buzzard that aims to reduce the renowned Fermat's Last Theorem to mathematical facts already known by mathematicians in the 1980s, a starting point similar to that of Andrew Wiles and Richard Taylor, who first proved this theorem in 1995 \cite{Buzzard2024}.

As mentioned above, one important concept that is currently missing in mathlib is CW-complexes. 
They were first invented by \Citeauthor{Whitehead2018} in 1949 in \cite{Whitehead2018} to state and prove the famous Whitehead theorem, which says that a continuous map between CW-complexes that induces isomorphisms on all homotopy groups is a homotopy equivalence.
CW-complexes are especially useful when doing calculations, for example, of singular homology and cohomology. 
One reason is that their skeletal structure allows one to use induction.
Since we are interested in providing a basic theory of CW-complexes, we will not focus on applications but instead on basic properties. 
An introduction to CW-complexes and their applications can be found in \cite{Lundell1969}.

Our mathematical discussion will mostly be based on \cite{Hatcher2001}.
In chapter 1 we will discuss CW-complexes from a purely mathematical perspective. 
Chapter 2 gives a short introduction to some aspects of Lean that will be useful to understand the formalisation of most of the content of chapter 1 which we will cover in chapter 3. 
Note that the focus of this thesis is the formalisation of CW-complexes. 
The accompanying code can be found at \url{https://github.com/scholzhannah/CWComplexes}.
Throughout this thesis we will link to code either from our formalisation or from mathlib. 
These links will be mark with this symbol : \faExternalLink.