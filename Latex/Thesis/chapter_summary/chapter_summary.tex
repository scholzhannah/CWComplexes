\chapter*{German summary}
\addcontentsline{toc}{chapter}{German summary}

Diese Arbeit befasst sich mit der Formalisierung von CW-Komplexen im Beweisassistenten Lean. 
Beweisassistenten können dazu genutzt werden, formal die Richtigkeit von Beweisen in einem logischen digitalen System zu überprüfen. 
Lean ist unter anderem wegen seiner umfangreichen mathematischen Bibliothek \emph{mathlib} ein sehr beliebter Beweisassistent. 
Ein Konzept, das in dieser Bibliothek jedoch noch fehlt, sind die CW-Komplexe. 
In der Topologie sind sie häufig hilreich, um Berechnungen, zum Beispiel von Homologie und Kohomologie, zu vereinfachen. 

Im ersten Kapitel beschäftigen wir uns mit der mathematischen Theorie hinter den CW-Komplexen. 
Wir konzentrieren uns hierbei auf die historische und nicht die moderne Definition, da uns diese die Formalisierung erleichtert. 
Wir beweisen einige grundlegende Eigenschaften von CW-Komplexen und beschäftigen uns dann im Detail mit verschiedenen Konstruktionen. 
Besonders dem Produkt zweier CW-Komplexes widmen wir sehr viel Zeit: Wir zeigen an einem Gegenbeispiel, dass das Produkt nicht notwendigerweise wieder ein CW-Komplex sein muss, führen dann k-Räume ein und beweisen, dass die k-ifizierung eines Produktes von zwei CW-Komplexen immer ein CW-Komplex ist.

Im zweiten Kapitel behandeln wir kurz drei technische Details von Lean: Die Typentheorie, d.h. die zugrundeliegende Logik, von Lean, implizite Argumente und Typklasseninferenz.
Diese Inhalte sind interessante Zusatzinformation, aber nicht unbedingt notwendig für das Verständnis der Arbeit. 

Im dritten Kapitel beschäftigen wir uns dann mit der Formalisierung von CW-Komplexen in Lean. 
Wir besprechen, welche Designentscheidungen getroffen wurden und warum, und machen auf Unterschiede in der Formalisierung aufmerksam. 
Dabei zeigen und erklären wir Ausschnitte des Codes. 
Wir haben einen Großteil des Inhalts des ersten Kapitels in Lean formalisiert, unter anderem den Zusammenhang zwischen endlichen Unterkomplexen und kompakten Mengen und die CW-Komplex-Struktur auf der k-ifizierung des Produktes von CW-Komplexen.
Den kompletten Code findet man unter \url{https://github.com/scholzhannah/CWComplexes}.