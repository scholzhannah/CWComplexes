\documentclass{beamer}
\usepackage[T1]{fontenc}
\usepackage[utf8]{inputenc}
\usepackage{listings}
\usepackage{amssymb}
\usepackage{upgreek}
\usepackage{mathtools}

\usepackage{color}
%colors for lean syntax
\definecolor{keywordcolor}{rgb}{0.7, 0.1, 0.1}   % red
\definecolor{tacticcolor}{rgb}{0.0, 0.1, 0.6}    % blue
\definecolor{commentcolor}{rgb}{0.4, 0.4, 0.4}   % grey
\definecolor{symbolcolor}{rgb}{0.0, 0.1, 0.6}    % blue
\definecolor{sortcolor}{rgb}{0.1, 0.5, 0.1}      % green
\definecolor{attributecolor}{rgb}{0.7, 0.1, 0.1} % red

\def\lstlanguagefiles{lstlean.tex}
% set default language
\lstset{language=lean}

\usetheme[compress]{Berlin}
\usecolortheme{default}

\newcommand{\bA}{\mathbb{A}}
\newcommand{\bB}{\mathbb{B}}
\newcommand{\bC}{\mathbb{C}}
\newcommand{\bD}{\mathbb{D}}
\newcommand{\bE}{\mathbb{E}}
\newcommand{\bF}{\mathbb{F}}
\newcommand{\bG}{\mathbb{G}}
\newcommand{\bH}{\mathbb{H}}
\newcommand{\bI}{\mathbb{I}}
\newcommand{\bJ}{\mathbb{J}}
\newcommand{\bK}{\mathbb{K}}
\newcommand{\bL}{\mathbb{L}}
\newcommand{\bM}{\mathbb{M}}
\newcommand{\bN}{\mathbb{N}}
\newcommand{\bO}{\mathbb{O}}
\newcommand{\bP}{\mathbb{P}}
\newcommand{\bQ}{\mathbb{Q}}
\newcommand{\bR}{\mathbb{R}}
\newcommand{\bS}{\mathbb{S}}
\newcommand{\bT}{\mathbb{T}}
\newcommand{\bU}{\mathbb{U}}
\newcommand{\bV}{\mathbb{V}}
\newcommand{\bW}{\mathbb{W}}
\newcommand{\bX}{\mathbb{X}}
\newcommand{\bY}{\mathbb{Y}}
\newcommand{\bZ}{\mathbb{Z}}

\newcommand{\fC}{\mathcal{C}}
\newcommand{\fL}{\mathcal{L}}
\newcommand{\fF}{\mathcal{F}}
\newcommand{\fR}{\mathcal{R}}
\newcommand{\fV}{\mathcal{V}}
\newcommand{\fM}{\mathcal{M}}
\newcommand{\fP}{\mathcal{P}}

\newcommand{\mathlib}{\texttt{Mathlib}\xspace}


\DeclareMathOperator{\aeq}{\equiv_{\alpha}} %alpha-equivalence
\newcommand{\alert}[1]{{\color{blue}{\relax\ifmmode\mathbf{#1}\else\textbf{#1}\fi}}}
\DeclareMathOperator{\cupdot}{\dot{\cup}} % Disjoint union
\DeclareMathOperator{\smodels}{\models_{\sigma}} %models with variable assignment
\DeclareMathOperator{\becont}{\triangleright_{\beta}} %beta contraction
\DeclareMathOperator{\bered}{\to_{\beta}} %beta reduction
\DeclareMathOperator{\beored}{\to_{\beta, 1}} %beta-one-reduction
\DeclareMathOperator{\beq}{\equiv_{\beta}} %beta-equivalence
\DeclareMathOperator{\ored}{\to_1} %one-reduction
\DeclareMathOperator{\iocont}{\triangleright_{\iota}} %iota contraction
\DeclareMathOperator{\iored}{\to_{\beta}} %iota reduction
\DeclareMathOperator{\ioored}{\to_{\beta, 1}} %iota-one-reduction
\DeclareMathOperator{\beiored}{\to_{\beta \iota}} %beta-iota reduction
\DeclareMathOperator{\beioored}{\to_{\beta \iota, 1}} %beta-iota-one-reduction


\newcommand{\pow}[1]{\fP(#1)} % power set
\newcommand{\fv}[1]{\text{fv}(#1)} %free variables
\newcommand{\abs}[1]{\lvert #1 \rvert} % abs value (underlying set of a model)
\newcommand{\interpret}[1]{\llbracket #1 \rrbracket} % interpretation in a model
\newcommand{\vect}[1]{\bar{#1}} % alternative vector line, not in use currently
\newcommand{\menquote}[1]{\ensuremath{\text{``} #1 \text{''}}} % quotes in math mode
\newcommand{\domain}[1]{\text{dom}(#1)} %domain
\newcommand{\down}{{\downarrow}} %downarrow without space around it
\newcommand{\up}{{\uparrow}} %uparrow without space around it
\newcommand{\suc}[1]{\text{Succ}(#1)} %successor function
\newcommand{\lamcur}{\lambda_{\to}^{\text{Curry}}} % Curry-style typed lambda calculus
\newcommand{\lamchu}{\lambda_{\to}^{\text{Church}}} % Church-style typed lambda calculus

\newcommand{\proj}[2]{\text{P}_{#1}^{#2}} %projection
\newcommand{\compl}[1]{#1^{c}} %complement
\newcommand{\id}{\text{id}} %identity
\newcommand{\preim}[2]{#1^{-1}(#2)} %preimage
\newcommand{\interior}[1]{\text{int}(#1)} %interior
\newcommand{\closure}[1]{\overline{#1}} %closure
\newcommand{\boundary}{\partial} %boundary
\newcommand{\restrict}[2]{\ensuremath{\left.#1\right|_{#2}}} %restriction
\newcommand{\norm}[1]{\left\lVert#1\right\rVert} %norm
\newcommand{\maximum}[2]{\text{max}(#1, #2)} %maximum




%Information to be included in the title page:
\title{Examples of CW-complexes}
\author{Hannah Scholz}
\institute[MI]{Mathematical Institute of the University of Bonn}
\date{28.01.2024}



\begin{document}

\frame{\titlepage}

\section{Introduction}

\begin{frame}
\frametitle{Definition of CW-complexes}
\fontsize{10pt}{5}\selectfont
  Let $X$ be a Hausdorff space.
    A \emph{CW-complex} on $X$ consists of a family of indexing sets $(I_n)_{n \in \bN}$ and a family of maps $(Q_i^n\colon D_i^n\rightarrow X)_{n \ge 0, i \in I_n}$ s.t.
    \setbeamertemplate{enumerate items}[default]
    \begin{enumerate}[(i)]
        \item $\restrict{Q_i^n}{\interior{D_i^n}}\colon \interior{D_i^n} \rightarrow Q_i^n(\interior{D_i^n})$ is a homeomorphism. We call $\openCell{n}{i} \coloneq Q_i^n(\interior{D_i^n})$ an \emph{(open) $n$-cell} (or a cell of dimension $n$)
        and $\closedCell{n}{i} \coloneq Q_i^n(D_i^n)$ a \emph{closed $n$-cell}.
        \item For all $n, m \in \bN$, $i \in I_n$ and $j \in I_m$ where $(n, i) \ne (m, j)$ the cells $\openCell{n}{i}$ and $\openCell{m}{j}$ are disjoint.
        \item For each $n \in \bN$, $i \in I_n$, $Q_i^n(\boundary D_i^n)$ is contained in the union of a finite number of closed cells of dimension less than $n$.
        \item $A \subseteq X$ is closed iff $Q_i^n(D_i^n) \cap A$ is closed for all $n \in \bN$ and $i \in I_n$.
        \item $\bigcup_{n \ge 0}\bigcup_{i \in I_n} Q_i^n(D_i^n) = X$.
    \end{enumerate}
    We call $Q_i^n$ a \emph{characteristic map} and $\cellFrontier{n}{i} \coloneq Q_i^n(\boundary D_i^n)$ the \emph{frontier of the $n$-cell} for any $i$ and $n$.
\end{frame}

\begin{frame}
  \frametitle{Examples of CW-complexes}
  We will look at the CW-complex structures on the following spaces: 
  \begin{itemize}
    \item $\varnothing$: The empty set. 
    \item Any finite set. 
    \item $[a, b]$: Any closed interval.
    \item $\mathbb{R}$: The real line. 
    \item $S^n$: The $n$-dimensional sphere.
  \end{itemize}
\end{frame}

\begin{frame}
  \frametitle{CW-complexes in Lean}
  \begin{itemize}
    \item I have defined and proven statements about CW-complexes as my bachelors thesis and as my work as a student research assistant.
    \item This version of CW-complexes is not (yet) in mathlib. 
    \item Proven statements include constructions like subcomplexes and products.
  \end{itemize}
\end{frame}

\section{Sphere: Direct version}

\begin{frame}
  \frametitle{Sphere (Direct version): Choosing a characteristic map}
  The main issue is how to define the characteristic map. 
  The three options I came up with are: 
  \setbeamertemplate{enumerate items}[default]
  \begin{itemize}
    \uncover<2->{\item Use spherical coordinates}
    \begin{itemize}
      \uncover<3->{\item[+] Easiest map}
      \uncover<4->{\item[-] Spherical coordinates are not in mathlib}
    \end{itemize}
    \uncover<5->{\item Define the map explicitly}
    \begin{itemize}
      \uncover<6->{\item[+] Only relies on basic calculation}
      \uncover<7->{\item[-] A lot of ugly calculations}
    \end{itemize}
    \uncover<8->{\item Compose the stereographic projection with a homeomorphism from $\mathbb{R}^n$ to $\interior{D^n}$}
    \begin{itemize}
      \uncover<9->{\item[+] All necessary properties on $\interior{D^n}$ are already in mathlib}
      \uncover<10->{\item[-] Showing continuity on $D^n$ is hard}
    \end{itemize}
  \end{itemize}
  \end{frame}

\begin{frame}
  \frametitle{Sphere (Direct version): Defining the characteristic map}
  Let $n \ge 2$, 
  
  $\text{\textbf{stereographic'}}: S^n \setminus \{p\} \to \mathbb{R}^n$ be the stereographic projection
  where $p$ is the north pole of the sphere and

  $\text{\textbf{unitBall}}: \mathbb{R}^n \to \interior{D^n}$ be the obvious map. 
  
  Then we define: 
  \begin{equation*}
  D^n \to S^n, x \mapsto 
  \begin{cases}
  (\text{unitBall} \circ \text{stereographic'})^{-1} x & x \in \interior{D^n} \\
  p & x \in S^{n - 1}
  \end{cases}
  \end{equation*}
\end{frame}

\begin{frame}
  \frametitle{Sphere (Direct version): Difficulties}
  \begin{itemize}
    \uncover<1->{\item I lost a lot of time trying the explicit version of the characteristic map.}
    \uncover<2->{\item The maps stereographic' and unitBall are wrappers for a compositions of a bunch of different functions and are designed for a specific purpose.}
    \uncover<3->{\item I don't really understand filters.}
  \end{itemize}
\end{frame}

\section{Sphere: Inductive version}

\begin{frame}
  \frametitle{Sphere (Inductive version): Choosing a characteristic map}
  I came up with the following two options: 
  \begin{itemize}
    \uncover<2->{\item Use the function \textbf{orthogonalProjection} from mathlib}
    \begin{itemize}
      \uncover<3->{\item[+] Possibly existing useful statements about the map}
      \uncover<4->{\item[-] Very general map, therefore hard to work with explicitly}
    \end{itemize}
    \uncover<5->{\item Define the function explicitly}
    \begin{itemize}
      \uncover<6->{\item[+] Only relies on basic calculations}
      \uncover<7->{\item[-] Need to describe properties of the map myself}
    \end{itemize}
  \end{itemize}
\end{frame}

\begin{frame}
  \frametitle{Sphere (Inductive Version): Difficulties}
  \begin{itemize}
    \uncover<1->{\item Construction is relatively technical, it was hard to write code that looked okay and wasn't too slow.}
    \uncover<2->{\item I didn't expect to have to do actual work for the inclusion.}
    \uncover<3->{\item Finding partial maps is hard.}
  \end{itemize}
\end{frame}

\section{Conclusion}

\begin{frame}
  \frametitle{Comparison the two definitions}
  \begin{itemize}
  \item Direct version
  \begin{itemize}
    \uncover<2->{\item[+] Simpler construction}
    \uncover<3->{\item[-] The characteristic map is considerably harder.}
    \uncover<4->{\item Spherical coordinated would probably eliminate that issue. The construction should be redone once they exist.}
  \end{itemize}
  \item Inductive version
  \begin{itemize}
    \uncover<5->{\item[+] Significantly easier characteristic maps}
    \uncover<6->{\item[-] More technical construction}
    \uncover<7->{\item[-] Would be a nightmare to unfold.}
  \end{itemize}
  \uncover<8->{\item I decided to set the direct version as the default}
  \end{itemize}
\end{frame}

\begin{frame}
  \frametitle{Future work}
  \begin{itemize}
    \item Generalize from unit spheres in euclidean space to all spheres under more metrics.
    \item Do more examples.
  \end{itemize}
\end{frame}

\end{document}