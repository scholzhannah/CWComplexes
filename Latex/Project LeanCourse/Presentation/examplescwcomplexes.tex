\documentclass{beamer}
\usepackage[T1]{fontenc}
\usepackage[utf8]{inputenc}
\usepackage{listings}
\usepackage{amssymb}
\usepackage{upgreek}
\usepackage{mathtools}

\usepackage{color}
%colors for lean syntax
\definecolor{keywordcolor}{rgb}{0.7, 0.1, 0.1}   % red
\definecolor{tacticcolor}{rgb}{0.0, 0.1, 0.6}    % blue
\definecolor{commentcolor}{rgb}{0.4, 0.4, 0.4}   % grey
\definecolor{symbolcolor}{rgb}{0.0, 0.1, 0.6}    % blue
\definecolor{sortcolor}{rgb}{0.1, 0.5, 0.1}      % green
\definecolor{attributecolor}{rgb}{0.7, 0.1, 0.1} % red

\def\lstlanguagefiles{lstlean.tex}
% set default language
\lstset{language=lean}

\usetheme{Madrid}
\usecolortheme{default}

\newcommand{\bA}{\mathbb{A}}
\newcommand{\bB}{\mathbb{B}}
\newcommand{\bC}{\mathbb{C}}
\newcommand{\bD}{\mathbb{D}}
\newcommand{\bE}{\mathbb{E}}
\newcommand{\bF}{\mathbb{F}}
\newcommand{\bG}{\mathbb{G}}
\newcommand{\bH}{\mathbb{H}}
\newcommand{\bI}{\mathbb{I}}
\newcommand{\bJ}{\mathbb{J}}
\newcommand{\bK}{\mathbb{K}}
\newcommand{\bL}{\mathbb{L}}
\newcommand{\bM}{\mathbb{M}}
\newcommand{\bN}{\mathbb{N}}
\newcommand{\bO}{\mathbb{O}}
\newcommand{\bP}{\mathbb{P}}
\newcommand{\bQ}{\mathbb{Q}}
\newcommand{\bR}{\mathbb{R}}
\newcommand{\bS}{\mathbb{S}}
\newcommand{\bT}{\mathbb{T}}
\newcommand{\bU}{\mathbb{U}}
\newcommand{\bV}{\mathbb{V}}
\newcommand{\bW}{\mathbb{W}}
\newcommand{\bX}{\mathbb{X}}
\newcommand{\bY}{\mathbb{Y}}
\newcommand{\bZ}{\mathbb{Z}}

\newcommand{\pr}[2]{\text{pr}_{#1}(#2)} %projection
\newcommand{\compl}[1]{#1^{c}} %complement
\newcommand{\id}{\text{id}} %identity

%Information to be included in the title page:
\title{Examples of CW-complexes}
\author{Hannah Scholz}
\institute[MI]{Mathematical Institute of the University of Bonn}
\date{23.01.2024}



\begin{document}

\frame{\titlepage}

\begin{frame}
\frametitle{Definition of CW-complexes}
  Let $X$ be a Hausdorff space.
    A \emph{CW-complex} on $X$ consists of a family of indexing sets $(I_n)_{n \in \bN}$ and a family of maps $(Q_i^n\colon D_i^n\rightarrow X)_{n \ge 0, i \in I_n}$ s.t.
    \setbeamertemplate{enumerate items}[default]
    \begin{enumerate}[(i)]
        \item $\restrict{Q_i^n}{\interior{D_i^n}}\colon \interior{D_i^n} \rightarrow Q_i^n(\interior{D_i^n})$ is a homeomorphism. We call $\openCell{n}{i} \coloneq Q_i^n(\interior{D_i^n})$ an \emph{(open) $n$-cell} (or a cell of dimension $n$)
        and $\closedCell{n}{i} \coloneq Q_i^n(D_i^n)$ a \emph{closed $n$-cell}.
        \item For all $n, m \in \bN$, $i \in I_n$ and $j \in I_m$ where $(n, i) \ne (m, j)$ the cells $\openCell{n}{i}$ and $\openCell{m}{j}$ are disjoint.
        \item For each $n \in \bN$, $i \in I_n$, $Q_i^n(\boundary D_i^n)$ is contained in the union of a finite number of closed cells of dimension less than $n$.
        \item $A \subseteq X$ is closed iff $Q_i^n(D_i^n) \cap A$ is closed for all $n \in \bN$ and $i \in I_n$.
        \item $\bigcup_{n \ge 0}\bigcup_{i \in I_n} Q_i^n(D_i^n) = X$.
    \end{enumerate}
    We call $Q_i^n$ a \emph{characteristic map} and $\cellFrontier{n}{i} \coloneq Q_i^n(\boundary D_i^n)$ the \emph{frontier of the $n$-cell} for any $i$ and $n$.
    Additionally we define $X_n \coloneq \bigcup_{m < n + 1} \bigcup_{i \in I_m} \closedCell{m}{i}$ and call it the \emph{$n$-skeleton} of $X$ for $-1 \le n \le \infty$.
\end{frame}

\begin{frame}
  \frametitle{Examples of CW-complexes}
  We will look at the CW-complex structures on the following spaces: 
  \begin{itemize}
    \item $\varnothing$: The empty set. 
    \item Any finite set. 
    \item $[a, b]$: Any closed interval.
    \item $\mathbb{R}$: The real line. 
    \item $S^n$: The $n$-dimensional sphere.
  \end{itemize}
\end{frame}

\begin{frame}
  \frametitle{CW-complexes in Lean}
  \begin{itemize}
    \item I have defined and proven statements about CW-complexes as my bachelors thesis and as my work as a student research assistant.
    \item CW-complexes are not (yet) in mathlib. 
    \item Proven statements include constructions like subcomplexes and products.
  \end{itemize}
\end{frame}

\begin{frame}
\frametitle{Equivalence of the definitions}
\begin{block}{Lemma}
  Let $C$ be a CW-complex in a Hausdorff space $X$ as in the definition in the formalisation.
  Then $C$ is a CW-complex as in the paper definition.
\end{block}
\begin{block}{Lemma}
  Let $X$ be a Hausdorff space and and $C$ a CW-complex in $X$ as in the formalised definition.
  Then $C$ is closed.
\end{block}
\end{frame}

\begin{frame}[fragile]{Closedness of CW-complexes}
  \begin{exampleblock}{Proof}
    \small
    \begin{tabular}{ll}
      1& By the weak topology it is enough to show that the intersection with every \\
      & closed cell is closed.\\
      2& Take any closed cell of $C$.\\
      3& Since the closed cell is a subset of $C$, the intersection is just the closed cell.\\
      4& Every closed cell is closed.\\
    \end{tabular}
  \end{exampleblock}
  \begin{exampleblock}{Proof (Lean)}
\begin{lstlisting}[basicstyle=\ttfamily\small, numbers=left, xleftmargin=21pt]
rw [closed _ (by rfl)]
intros
rw [inter_eq_right.2 (closedCell_subset_complex _ _)]
exact isClosed_closedCell
\end{lstlisting}
  \end{exampleblock}

\end{frame}

\begin{frame}{The topology of the product}
\begin{block}{Definition k-space}
  Let $X$ be a topological space. 
  We call $X$ a \emph{k-space} if 
  \begin{align*}
    A \subseteq X \text{ is closed} \iff &\text{for all compact sets } C \subseteq X \text{ the intersection } A \cap C \\
    &\text{is closed in } C.
  \end{align*}
\end{block}
\begin{block}{Lemma}
  Let $X$ be a CW-complex and $C \subseteq X$ a compact set. 
  Then $C$ is disjoint with all but finitely many cells of $X$.
\end{block}
\begin{block}{Lemma}
  If $X \times Y$ is a k-space then it has weak topology with respect to the characteristic maps $(Q_i^n \times P_j^m \colon D_i^n \times D_j^m \to (X \times Y)_c)_{n,m \in \bN,i \in I_n,j \in J_m}$, i.e. $A \subseteq X \times Y$ is closed iff $A \cap (\closedCell{n}{i} \times \closedCellf{m}{j})$ is closed for every $n, m \in \bN$, $i \in I_n$ and $j \in J_m$.
\end{block}
\end{frame}

\end{document}