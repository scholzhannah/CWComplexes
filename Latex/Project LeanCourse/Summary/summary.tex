\documentclass[11pt,a4paper]{article}
\usepackage[utf8]{inputenc}
%\usepackage{ngerman}
%\usepackage{latexsym}
\usepackage{amsmath}
\usepackage{amssymb,graphicx}
%\usepackage{pxfonts} 
\parindent=0cm
\topmargin -2.5cm
\oddsidemargin -0cm
\textwidth 16cm
\textheight 26cm 
\pagestyle{empty}
\nofiles

\def\C{\mathbb{C}}
\def\R{\mathbb{R}}
\def\K{\mathbb{K}}
\def\Q{\mathbb{Q}}
\def\Z{\mathbb{Z}}
\def\N{\mathbb{N}}
\def\H{\mathbb{H}}
\def\e{\varepsilon}
\newcommand{\dif}[1]{\,\mathrm{d} #1}
\newcommand{\norm}[1]{\lVert #1 \rVert}
\newcommand{\vii}[2]{\ensuremath{\begin{pmatrix}#1 \\ #2 \end{pmatrix}}}
\newcommand{\mii}[4]{\ensuremath{\begin{pmatrix}#1&#2 \\ #3&#4 \end{pmatrix}}}
\DeclareMathOperator{\grad}{grad}

\begin{document}

\begin{minipage}{0.75\linewidth}
{
  \Large \bfseries Summary Lean Project "Examples of CW-complexes"\\[0.5ex]
  \mdseries
  \normalsize for the class "Practical Project in Mathematical Logic - Formalized Mathematics in Lean" 
    taught by Prof. van Doorn \\[0.4ex]
  \normalsize 25.01.2025\\[0.4ex]
  \normalsize Hannah Scholz
}
\end{minipage}
\hfill
\begin{minipage}{0.25\linewidth}
\vspace{2mm}
\end{minipage}


\bigskip 
{\bf Description}

\medskip

In this project I have formalized a few examples of CW-complexes in Lean. 
The corresponding files can be found in this repository in the folder CWComplexes/Project.
The code relies on my formalization of CW-complexes which can be found in the folder CWComplexes in the repository.

In addition to all the contents that were requested of this file, I will also try to make clear
which work was done entirely for this project and which work was done in part for my thesis or my work as 
a student assistant. 
I have not declared the hours spent on this project as working hours. 

The examples I have formalized are: The empty set, any finite set, any closed interval in the reals, the reals, and 
two different constructions on the unit sphere in $n$-dimensional euclidean space. 

I did not use any particular reference for this project. I think these examples count as common knowledge in topology.
However, the suggestion for the characteristic map of the construction on the sphere with two cells overall 
came from Prof. van Doorn. 

In the the following two sections I will discuss the contents of all of the files and talk about work that is left to do.
The main part of this project is part of the files \emph{SpheresAux} and \emph{Spheres}.


\bigskip

{\bf Contents by file}

\medskip

\begin{itemize}
    \item \textbf{Examples}

This file covers all the examples except for the spheres.

I had already written a previous version of this file for my thesis.
That previous version contained the following constructions: 
The empty set, any singleton, and any interval.
So this work is not entirely new. 
If you want to see what the previous work was exactly, 
you can look at the code release corresponding to my thesis. 
All work after that is a part of this project.

As a part of this project I generalized the construction of the singleton to all finite sets, 
I improved the construction of the interval by changing the definition 
(using the construction \emph{ClasCWComplex.attachCellFiniteType} instead of the constructor for CW-complexes), 
and I added the construction of the reals.

The most important results are: 

\emph{instEmpty}: The empty set is a CW-complex.

\emph{instFiniteSet}: Every finite set is a CW-complex.

\emph{instIcc}: The interval $[a, b]$ in $\R$ is a CW-complex.

\emph{instReal}: The real numbers are a CW-complex.

    \item \textbf{Homeomorph}

In this file I proved that a partial bijection with some nice properties preserves CW-complexes. 
This is a generalization of the statement \emph{RelCWComplex.ofHomeomorph} that can be found in the 
file \emph{RelConstructions}. 
I had tried to prove this new version before starting this project but had not succeeded. 
Since I needed it for the second construction on spheres, 
I improved upon my broken version for this project. 

The most important result from this file is: 

\emph{RelCWComplex.ofPartialEquiv}: A partial bijection with closed source and target that is
continuous on both source and target preserves CW-structures.

    \item \textbf{SpheresAux}

The contents of this file were written entirely for this project. 
This file contains the preliminary work for the construction of the spheres. 
As such it actually contains the bulk of the work for the first construction on the sphere. 

The main results from this file are: 

\emph{Homeomorph.tendsto\_norm\_comp\_unitBall\_symm}: As we approach the sphere from inside the ball the
inverse of \emph{Homeomorph.unitBall} tends to infinity in its norm.

\emph{stereographic'\_symm\_tendsto}: As we approach infinite norm the inverse of the stereographic
projection \emph{stereographic'} approaches the centre of the projection.

\emph{normScale}: A homeomorphism from one normed group to another that preserves norms and the zero.

    \item \textbf{Spheres}

This file contains the actual constructions on the spheres. 
The content of this file was also written entirely for this project. 
The first construction contained in this file is the CW-structure on the sphere that uses two cells in 
total. 
As mentioned before this construction relies heavily on some of the contents of \emph{SpheresAux}. 

Afterwards, you can find the construction on the sphere with two cells in each dimension. 
This construction uses the contents of the file \emph{Homeomorph} and also some of the content in 
\emph{SpheresAux}. 
Still, the most difficult part of this construction is contained in this file. 

The main result from this file are: 

\emph{instSphere}: The unit sphere is a CW-complex. This construction uses two cells in total.

\emph{SphereInduct}: The unit sphere is a CW-complex. This construction uses two cells in each
dimension.

\end{itemize}

\bigskip

{\bf Unfinished work}

\medskip

The project does not contain any 'sorry's and also does not make use of any part of my previous work 
that contains 'sorry's. 

The code does however contain some 'ToDo's. 
The most important thing that is missing is a generalization of the constructions on spheres. 
Currently, I have only shown the statements for unit spheres in euclidean space. 
The results should be generalized to other center points, radii and metrics. 
(I hope that this isn't actually too hard.)
I will probably do that in the future. 


\end{document}
