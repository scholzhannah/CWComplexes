\documentclass{article}
\usepackage[T1]{fontenc}
\usepackage[utf8]{inputenc}
\usepackage{listings}
\usepackage{amssymb}
\usepackage{upgreek}
\usepackage{mathtools}
\usepackage{amsthm}


\newcommand{\bA}{\mathbb{A}}
\newcommand{\bB}{\mathbb{B}}
\newcommand{\bC}{\mathbb{C}}
\newcommand{\bD}{\mathbb{D}}
\newcommand{\bE}{\mathbb{E}}
\newcommand{\bF}{\mathbb{F}}
\newcommand{\bG}{\mathbb{G}}
\newcommand{\bH}{\mathbb{H}}
\newcommand{\bI}{\mathbb{I}}
\newcommand{\bJ}{\mathbb{J}}
\newcommand{\bK}{\mathbb{K}}
\newcommand{\bL}{\mathbb{L}}
\newcommand{\bM}{\mathbb{M}}
\newcommand{\bN}{\mathbb{N}}
\newcommand{\bO}{\mathbb{O}}
\newcommand{\bP}{\mathbb{P}}
\newcommand{\bQ}{\mathbb{Q}}
\newcommand{\bR}{\mathbb{R}}
\newcommand{\bS}{\mathbb{S}}
\newcommand{\bT}{\mathbb{T}}
\newcommand{\bU}{\mathbb{U}}
\newcommand{\bV}{\mathbb{V}}
\newcommand{\bW}{\mathbb{W}}
\newcommand{\bX}{\mathbb{X}}
\newcommand{\bY}{\mathbb{Y}}
\newcommand{\bZ}{\mathbb{Z}}

\newcommand{\fC}{\mathcal{C}}
\newcommand{\fL}{\mathcal{L}}
\newcommand{\fF}{\mathcal{F}}
\newcommand{\fR}{\mathcal{R}}
\newcommand{\fV}{\mathcal{V}}
\newcommand{\fM}{\mathcal{M}}
\newcommand{\fP}{\mathcal{P}}

\newcommand{\mathlib}{\texttt{Mathlib}\xspace}


\DeclareMathOperator{\aeq}{\equiv_{\alpha}} %alpha-equivalence
\newcommand{\alert}[1]{{\color{blue}{\relax\ifmmode\mathbf{#1}\else\textbf{#1}\fi}}}
\DeclareMathOperator{\cupdot}{\dot{\cup}} % Disjoint union
\DeclareMathOperator{\smodels}{\models_{\sigma}} %models with variable assignment
\DeclareMathOperator{\becont}{\triangleright_{\beta}} %beta contraction
\DeclareMathOperator{\bered}{\to_{\beta}} %beta reduction
\DeclareMathOperator{\beored}{\to_{\beta, 1}} %beta-one-reduction
\DeclareMathOperator{\beq}{\equiv_{\beta}} %beta-equivalence
\DeclareMathOperator{\ored}{\to_1} %one-reduction
\DeclareMathOperator{\iocont}{\triangleright_{\iota}} %iota contraction
\DeclareMathOperator{\iored}{\to_{\beta}} %iota reduction
\DeclareMathOperator{\ioored}{\to_{\beta, 1}} %iota-one-reduction
\DeclareMathOperator{\beiored}{\to_{\beta \iota}} %beta-iota reduction
\DeclareMathOperator{\beioored}{\to_{\beta \iota, 1}} %beta-iota-one-reduction


\newcommand{\pow}[1]{\fP(#1)} % power set
\newcommand{\fv}[1]{\text{fv}(#1)} %free variables
\newcommand{\abs}[1]{\lvert #1 \rvert} % abs value (underlying set of a model)
\newcommand{\interpret}[1]{\llbracket #1 \rrbracket} % interpretation in a model
\newcommand{\vect}[1]{\bar{#1}} % alternative vector line, not in use currently
\newcommand{\menquote}[1]{\ensuremath{\text{``} #1 \text{''}}} % quotes in math mode
\newcommand{\domain}[1]{\text{dom}(#1)} %domain
\newcommand{\down}{{\downarrow}} %downarrow without space around it
\newcommand{\up}{{\uparrow}} %uparrow without space around it
\newcommand{\suc}[1]{\text{Succ}(#1)} %successor function
\newcommand{\lamcur}{\lambda_{\to}^{\text{Curry}}} % Curry-style typed lambda calculus
\newcommand{\lamchu}{\lambda_{\to}^{\text{Church}}} % Church-style typed lambda calculus

\newcommand{\proj}[2]{\text{P}_{#1}^{#2}} %projection
\newcommand{\compl}[1]{#1^{c}} %complement
\newcommand{\id}{\text{id}} %identity
\newcommand{\preim}[2]{#1^{-1}(#2)} %preimage
\newcommand{\interior}[1]{\text{int}(#1)} %interior
\newcommand{\closure}[1]{\overline{#1}} %closure
\newcommand{\boundary}{\partial} %boundary
\newcommand{\restrict}[2]{\ensuremath{\left.#1\right|_{#2}}} %restriction
\newcommand{\norm}[1]{\left\lVert#1\right\rVert} %norm
\newcommand{\maximum}[2]{\text{max}(#1, #2)} %maximum




\newtheorem{theorem}{Theorem}[section]
\newtheorem{corollary}{Corollary}[theorem]
\newtheorem{lemma}[theorem]{Lemma}

\begin{document}

\begin{lemma}
    Let $C$ be a CW-complex in a Hausdorff space $X$ as in the definition in the formalisation.
    Then $C$ is a CW-complex as in the paper definition.
  \end{lemma}
  \begin{proof}
    Properties (i), (ii), (iii) and (v) of the definition are immediate.
    Thus let us look at property (iv).
    We assume that
    \[A \subseteq C \text{ is closed in } X \iff \closedCell{n}{i} \cap A \text{ is closed in } X \text{ for all } n \in \bN \text{ and } i \in I_n\]
    and need to show that
    \[A \subseteq C \text{ is closed in } C \iff \closedCell{n}{i} \cap A \text{ is closed in } C \text{ for all } n \in \bN \text{ and } i \in I_n.\]
    It is easy to see that the forward direction is true.
    For the backwards direction take $A \subseteq C$ such that $A \cap \closedCell{n}{i}$ is closed in $C$ for all $n \in \bN$ and $i \in I_n$.
    That means that for every $n \in \bN$ and $i \in I_n$ there is a closed set $B_i^n \subseteq X$ such that $B^n_i \cap C = A \cap \closedCell{n}{i}$.
    But since $C$ is closed that means that $A \cap \closedCell{n}{i}$ was already closed for every $n \in \bN$ and $i \in I_n$.
    Thus we are done by assumption.
  \end{proof}


  \begin{lemma}\label{lem:weaktopologyproduct}
    If $X \times Y$ is a k-space then it has weak topology,
    i.e. $A \subseteq X \times Y$ is closed iff $\closedCell{n}{i} \times \closedCellf{m}{j} \cap A$ is closed for all $n, m \in \bN$, $i \in I_n$ and $j \in J_m$. 
\end{lemma}
\begin{proof}
    The forward direction is easy.

    Moving on to the backward direction we know that $A$ is closed if for every compact set $C \subseteq (X \times Y)_c$, $A \cap C$ is closed in $C$.
    Take such a compact set $C$.
    The projections $\pr{1}{C}$ and $\pr{2}{C}$ are compact as images of a compact set. 
    There are finite sets $E \subseteq \{e_i^n \mid n \in \bN, i \in I_n \}$ and $F \subseteq \{f_j^m \mid m \in \bN, j \in J_m \}$ s.t $\pr{1}{C} \subseteq \bigcup_{e \in E} e$ and $\pr{2}{C} \subseteq \bigcup_{f \in F} f$.
    Thus 
    \[C \subseteq \pr{1}{C} \times \pr{2}{C} \subseteq \bigcup_{e \in E} e \times \bigcup_{f \in F} f = \bigcup_{e \in E} \bigcup_{f \in F} e \times f.\] 
    So $C$ is included in a finite union of cells of $(X \times Y)_c$. 
    Therefore 
    \[A \cap C = A \cap \left (\bigcup_{e \in E} \bigcup_{f \in F} e \times f \right )\cap C = \left (\bigcup_{e \in E} \bigcup_{f \in F} A \cap (e \times f)\right ) \cap C\] 
    is closed since by assumption $A \cap (e \times f)$ is closed for every $e$ and $f$ and the union is finite. Thus $A \cap C$ is in particular closed in $C$.
\end{proof}

\end{document}